% -*- mode: latex; coding: utf-8-unix; TeX-PDF-mode: t; fill-column: 78 -*-
\documentclass[a4paper,11pt]{article}

%Enconding
\usepackage[utf8]{inputenc} % enconding de entrada, desde 2018 por defecto usa UTF8
\usepackage[T1]{fontenc} % enconding de salida (en PdfLaTeX, para XeLaTeX se recomienda \usepackage{fontenc})
\usepackage[english,spanish]{babel} %Se usan ambas configuraciones de idiomas

%Texto
\usepackage{mathptmx} %Usa Times como fuente de texto, y provee soporte matemático
\usepackage{helvet} %Fuente de texto Arial, y guarda cierta compatibilidad con Times
\usepackage{enumerate} %Para enumerar listas

\usepackage{url} %Para usar enlaces web
\usepackage{hyperref}  %\usepackage[pdftex]{hyperref} %si hay problemas al compilar con PdfLaTeX
\usepackage[natbibapa]{apacite} %Agregar formato de citación APA

%Matemática AMS (no usados)
%\usepackage{amsmath}
%\usepackage{amssymb}
%\usepackage{amsthm}


\parskip=3pt\clubpenalty=10000\widowpenalty=10000
\footnotesep=\baselineskip
\addtocounter{tocdepth}{-1}

\def\btx-{\textsc{Bib\TeX}}
\def\ltx-{\LaTeX}
\def\ltr#1-{<<\texttt{#1}>>}
\def\tpf#1-{\ltr.#1-}
\def\cmd#1-{``\texttt{\textbackslash#1}''}
\newcommand{\flecha}{\ensuremath{\rightarrow}}


\hypersetup{
  colorlinks = true,
  pdftitle = {Guía casi completa de BibTeX},
  pdfauthor = {Joaquin Ataz López},
  pdfsubject = {Uso de BibTeX para la gestión de las referencias bibliográficas en LaTeX.},
  pdfkeywords = {LaTeX BibTeX},
}


\begin{document}

\renewcommand{\thefootnote}{\fnsymbol{footnote}}

\title{Guía \emph{básica} de \btx-\footnotemark}

\footnotetext[1]{Copyright \copyright{} 2023. Daniel Camarena.\\
  \selectlanguage{english} Permisssion  is granted to  copy, distribute and/or
  modify this document under the  terms of the GNU Free Documentation License,
  Version 1.3 or  any later version published by  de Free Software Foundation;
  with no Invariant Sections, no Front-Cover Texts, and no Back-Cover Texts. A
  Copy  of  the  license  is  included  in the  section  entitled  ``GNU  Free
  Documentation  License''.\\\selectlanguage{spanish} \emph{Se  otorga permiso
    para copiar, distribuir  o modificar este documento en  los términos de la
    Licencia  GNU para Documentación  Libre, versión  1.3 o  cualquier versión
    posterior  publicada  por  la   Free  Software  Foundation;  sin  secciones
    invariantes, sin textos de la cubierta frontal y sin textos de la cubierta
    posterior. Una copia completa de la  licencia (en inglés) se incluye en el
    apéndice titulado ``GNU Free Documentation License''}. }


\author{Daniel Camarena}

\date{Versión  1,  \today\\\footnotesize   Para  sugerencias  y  correcciones:
  \texttt{vcamarenap@uni.pe}}

\maketitle

\renewcommand{\thefootnote}{\arabic{footnote}}

\begin{abstract}

  \btx- es un programa auxiliar de \ltx-, diseñado para facilitar el manejo de
  la bibliografía. Pero es más que eso: es también una herramienta que permite
  a  \ltx- extraer  datos de  una base  de datos  e  insertarlos adecuadamente
  formateados en un documento.

  Este  texto pretende  abordar todos  los aspectos  de \btx-,  desde  los más
  básicos   a   los   más   complejos,   haciendo  además   hincapié   en   su
  \emph{españolización}, pues \btx-, como tantas otras herramientas, presupone
  que será usada para trabajar con documentos en inglés.

\end{abstract}

{\parskip=0pt\small\tableofcontents}

\section*{Preliminar}
\label{sec:preliminar}

\addcontentsline{toc}{section}{Preliminar}

\btx- es un  programa que amplia notablemente las capacidades  de \ltx- con el
manejo de  la bibliografía. Las  grandes ventajas que ofrece  su uso\footnote{
  \btx- nos ayuda  a gestionar nuestra bibliografía de  manera independiente a
  los documentos en  los que debe aparecer, al tiempo  que nos permite generar
  listas  de referencias  bibliográficas formateadas  de manera  consistente y
  cambiar  fácilmente  determinadas  características  de las  mismas.   }   se
incrementan cuando  se maneja una lista  bibliográfica amplia y  cuando con la
misma bibliografía hay que escribir varios documentos.

No hay  mucha documentación sobre \btx-  en español. Publicadas  en papel sólo
conozco dos obras, ambas de los mismos autores. La primera, \cite{cascales00},
sin duda constituye la más completa información sobre \btx- en nuestro idioma:
mas de 40  páginas en las que se incluyen incluso  los aspectos más avanzados.
La segunda,  \cite{cascales03}, no es  tan extensa, pero constituye  una buena
introducción en la que se cubren todos los aspectos básicos y algunos aspectos
avanzados.

En formato electrónico  y en nuestro idioma conozco  otros dos documentos. Uno
de  Luis Seidel,  \cite{seidel98} que  es  una buena  introducción, aunque  no
resulta muy  extenso.  Otro  de José Manuel  Mira, \cite{mira04}, que  no está
pensado para quienes se aproximen por  primera vez a esta herramienta, pues en
él  se explica un  \emph{paquete} (\ltr  flexbib-) cuyo  uso presupone  que ya
conocemos lo básico del trabajo con \btx-.

Como es lógico, en inglés  hay mucha más documentación. Los textos principales
son el  apéndice B del Libro  sobre \ltx- de Lamport,  \cite{lamport86}, y los
dos   documentos  que  sobre   \btx-  escribió   su  autor,   Oren  Patashnik:
\cite{patashnik88} y \cite{patashnik88-2}. El primero se ocupa de los aspectos
más  básicos,  y está  concebido  como  una  \emph{corrección} del  mencionado
apéndice B, y el segundo se  ocupa de los aspectos más avanzados.  El problema
que  tienen estos  documentos  es que  el  trabajo de  Lamport  es difícil  de
localizar para el público español.

Además  de los documentos  anteriores, en  Internet pueden  localizarse varios
textos, la  mayoría en  inglés, algunos introductorios  y otros no  tanto.  De
entre  ellos  me parecen  destacables  \cite{young02}  (un documento  bastante
equilibrado) y  \cite{markey05}, dirigido  a un público  más avanzado  que los
otros textos mencionados.

\begin{center}***\end{center}

El  objetivo  fundamental  de  este  documento  es  el  de,  partiendo  de  la
documentación  que se  acaba de  citar, salvo  \cite{lamport86} al  que  no he
tenido  acceso,  y  alguna  otra   que  no  se  ha  mencionado  todavía,  como
\cite{kopla}, poner al alcance del  público hispanoparlante un documento en el
que se aborden  todos los aspectos de  \btx-, desde los más básicos  a los más
avanzados. De ahí el  título que he elegido, que tal vez  a alguien le suene a
pretencioso.  Desde luego esta guía  no es completa, porque hay aspectos sobre
los que se  pasa casi de puntillas\footnote{Como por  ejemplo las herramientas
  para gestionar ficheros \tpf  bib-, numerosos estilos bastante extendidos, o
  los paquetes de  ampliación para trabajar con \btx-.}.   Tal vez ni siquiera
sea   \emph{casi}    completa.    Pero   sí    es   la   más    completa   que
conozco\footnote{Quizás  con  la  salvedad  de \cite{cascales00}.   Aunque  el
  problema de ese libro es que hoy no es ya fácil de localizar.}.

La materia a tratar ha sido  distribuida en tres partes, atendiendo a su nivel
de complejidad y detalle:

\begin{enumerate}

\item \textbf{\btx- básico:} Cubre  los aspectos básicos del funcionamiento de
  \btx- que  afectan a las instrucciones  que hay que incluir  en el documento
  principal de \ltx-, al  diseño y mantenimiento de la base de  datos, y a los
  distintos estilos  estándar existentes, así  como a los  procedimientos para
  \emph{españolizarlos}. Es la parte más  extensa del documento. Se supone que
  sólo con leer  esta parte estaremos totalmente preparados  para trabajar con
  \btx-.

\item \textbf{\btx-  intermedio:} En esta  parte se describen aspectos  que no
  son  imprescindibles para  trabajar con  \btx-, pero  cuyo  conocimiento nos
  proporcionará un  mayor control. En particular  se aborda el  uso de makebst
  para generar nuestro propio fichero de estilo.

\item \textbf{\btx- avanzado:} Esta última parte entra a fondo en los ficheros
  de estilo de \btx-. En ella  se describe el lenguaje que estos ficheros usan
  internamente.

\end{enumerate}

Para entender este documento  no hay que ser un La\TeX perto,  pero sí hay que
saber  lo suficiente  de \ltx-,  y  del procedimiento  estándar que  en él  se
implementa para trabajar con referencias bibliográficas.

En general el  contenido de este documento valdrá  para cualquier distribución
de  \ltx- que  incluya  \btx- (o  sea,  todas las  distribuciones).  Pero  hay
algunos aspectos que dependen de la concreta instalación que se haya hecho del
sistema como, por ejemplo, la localización de los directorios en los que \btx-
lee  las  bases  de datos  o  los  ficheros  de  estilo.   En tales  casos  la
explicación que doy  no vale para cualquier sistema,  sino exclusivamente para
el mío,  que es una  instalación estándar de  \ltx- hecha en un  sistema donde
funciona Ubuntu versión 6.06.  Aunque, en general, estos aspectos dependientes
del concreto sistema utilizado son tratados en nota a pie de página.

Asimismo, a lo largo del presente documento se citan en ocasiones determinados
paquetes para \ltx-  o ficheros con estilos adicionales  para \btx-. Salvo que
se diga lo  contrario, todos los paquetes y  estilos citados están disponibles
en la  CTAN, (<<\emph{Comprehensive TeX Archives Network}>>)  y deben buscarse
en:

\begin{itemize}

\item <<\url{http://www.ctan.org/tex-archive/macros/latex/contrib/}>>

\item <<\url{http://www.ctan.org/tex-archive/biblio/bibtex/contrib/}>>

\end{itemize}

\clearpage

\part{\btx- básico}
\label{part:btx-basico}

\section{Dinámica general del sistema}
\label{sec:entender-la-dinamica}

\subsection{Preparar nuestro documento principal para usar \btx-}
\label{sec:prep-nuestro-docum}


\paragraph{Las órdenes bibliography y bibliographystyle:\\[\baselineskip]}
\label{sec:las-orden-bibl}

Para usar \btx- necesitamos dos cosas:

\begin{enumerate}

\item Tener  almacenadas en un fichero aparte,  las referencias bibliográficas
  que pensemos usar.

\item En el  lugar del documento principal en el que  queramos que aparezca la
  lista con  las referencias  bibliográficas, debemos insertar  las siguientes
  dos órdenes de \ltx-:

  {\small
\begin{verbatim}
\bibliography{MiBiblio}
\bibliographystyle{MiEstilo}
\end{verbatim}
  }

  Estas  órdenes  provocan que  tras  las  oportunas  compilaciones (véase  la
  sección \ref{sec:como-generar-la}, página \pageref{sec:como-generar-la}), se
  genere una lista  con las referencias bibliográficas usadas  en el documento
  en la  que, además, a cada una  de ellas se le  asignará una \emph{etiqueta}
  identificativa\footnote{\textbf{NOTA TERMINOLÓGICA:}  En la documentación en
    castellano  de  \ltx-  el  término  etiqueta  se  suele  usar  para  hacer
    referencia a  la cadena de texto que  usan en las órdenes  que sirven para
    tratar con referencias  cruzadas tales como \cmd cite-,  \cmd label-, \cmd
    index-  o  \cmd  glosary-.   Este  tipo de  \emph{etiquetas}  son  de  uso
    puramente interno, es decir: \ltx-  las usa como marcadores para apuntar a
    distintas partes del documento, pero no  se imprimen en él.  Junto a estas
    \emph{etiquetas}  también se  usa  a  veces el  mismo  término para  hacer
    referencia al rótulo identificador de  una entrada concreta de la lista de
    referencias bibliográficas,  que se imprime en el  documento.  Para evitar
    confusiones,  en este  documento he  reservado el  término \emph{etiqueta}
    para las que llegan a imprimirse  en el documento final, mientras que para
    referirme  a las  usadas  internamente como  marcadores,  he reservado  el
    término \emph{clave}.  }.  Los  datos correspondientes a tales referencias
  serán buscados en el fichero \ltr MiBiblio.bib-, y la lista se formateará de
  acuerdo con  las indicaciones de  estilo que se  contengan en el  fichero de
  estilo \ltr MiEstilo.bst- y se insertará en el lugar del documento en el que
  se encuentre  \cmd bibliography-.   Las instrucciones de  estilo controlarán
  asimismo  el tipo de  etiqueta identificativa  que se  asignará a  las obras
  incluidas en  la lista de referencias. Si  queremos usar más de  una base de
  datos  podemos  indicarlas, separadas  por  comas,  como  argumento de  \cmd
  bibliography-.  El  efecto de  esto último será  que las distintas  bases de
  datos   indicadas   se   concatenerán   en   el   orden   en   el   que   se
  indicaron\footnote{Lo  que tiene  importancia, porque  en los  ficheros \tpf
    bib-  hay utilidades  que sólo  funcionan si  el orden  en el  que ciertos
    elementos se encuentran  dentro del fichero es el  correcto.  Por ejemplo,
    las abreviaturas deben estar definidas antes de ser usadas, de modo que si
    tenemos un fichero  que recoja una serie de abreviaturas  para \btx- y una
    base  de datos  que  las use,  siempre  habrá que  indicar  el fichero  de
    abreviaturas antes que la base de datos.}.

\end{enumerate}

\subsubsection*{Nombres de ficheros y directorios:}
\label{sec:nombres-de-ficheros}

El  argumento de  \cmd bibliography-  y el  de \cmd  bibliographystyle-  es el
nombre  de un  fichero.   Ambos  nombres deben  indicarse  sin especificar  la
extensión, pues se asume  que esta será la asignada al tipo  de fichero de que
se trate.   De hecho \btx-  no funciona correctamente  cuando el nombre  de la
base de  datos no  tiene la extensión  \tpf bib-,  ni tampoco cuando  en dicho
nombre existe un espacio en blanco  o algún carácter no anglosajón como eñes o
vocales acentuadas\footnote{Así ocurre en mi  sistema. Es posible que en otros
  sistemas operativos  no exista  este problema, aunque  creo que  ocurrirá lo
  mismo, al menos  en lo relativo a la extensión de  los ficheros; porque \cmd
  thebibliography- se  limita a copiar  en el fichero  \tpf aux- el  texto que
  recibe como  argumento, y  \btx- le  añade a dicho  texto la  extensión \tpf
  bib-, salvo en el caso de que ya termine así.}.

Los  ficheros, el  de bibliografía  y el  de estilo,  deben encontrarse  en el
directorio  de  trabajo  (el  mismo  en  el  que  se  encuentre  el  documento
principal), o en alguno de los directorios en los que \ltx- busca por defecto,
los cuales varían  dependiendo de la distribución de \ltx-  concreta de que se
disponga\footnote{\label{nota:cnf}En  \ltr web2c-, que  es la  distribución de
  \ltx- estándar para  sistemas Unix/Linux, el árbol de  ficheros de \TeX{} se
  encuentra  en \ltr  /usr/share/texmf- Allí  hay un  fichero  denominado \ltr
  texmf.cnf- en el  que se almacenan las variables que  controlan las rutas de
  búsqueda por  defecto de \ltx- y  sus programas auxiliares.   Aunque a veces
  dicho  fichero  se  encuentra  en  el  directorio \ltr  web2c-,  o  en  \ltr
  /etc/texmf/-.  En dicho fichero se  encuentran las dos variables que afectan
  a los  directorios relativos a \btx-:  BIBINPUTS (directorios en  los que se
  buscarán  los ficheros \tpf  bib-) y  BSTINPUTS (directorios  en los  que se
  buscarán los ficheros de estilo).  }.   Si los ficheros que queremos usar no
están en  alguno de estos directorios  deberemos incluir, junto  con el nombre
del fichero,  la ruta  de acceso  al mismo, en  cuyo caso  para dicha  ruta se
aplican  las mismas  reglas  que para  el  nombre de  los  ficheros: no  podrá
contener espacios en blanco ni caracteres no anglosajones.

\paragraph{Contenido de la lista de referencias a generar:\\[\baselineskip]}
\label{sec:contenido-de-la}

Las características \emph{formales} y  de ordenación de la lista bibliográfica
que  se generará,  dependen del  estilo concreto  usado, y  se explican  en la
sección \ref{sec:los-estil-bibl}.   Aquí me concentraré  exclusivamente en los
siguientes dos aspectos de la misma:

\begin{description}

\item[Qué referencias  se incluyen en la  lista:] En ella  se contendrán todas
  las  referencias bibliográficas  que, a  lo largo  de nuestro  documento, se
  hayan referenciado mediante los comandos \cmd cite- y \cmd nocite-:

  \begin{itemize}

  \item  El  comando \cmd  cite[DatosAdicionales]\{clave\}-  produce un  doble
    efecto. En  primer lugar, la  referencia bibliográfica identificada  en la
    base de datos mediante la clave recibida como parámetro, se incluirá en la
    lista bibliográfica. En segundo lugar,  en el punto del documento donde se
    encontrara el comando, se imprimirá  la etiqueta asignada a tal referencia
    en  la  lista   de  referencias  junto  con  los   datos  adicionales  que
    eventualmente hayamos incluido en el argumento opcional del comando.

  \item El  comando \cmd  nocite\{clave\}- produce el  primero de  los efectos
    indicados, pero no el segundo, es decir: la obra identificada por la clave
    será incluida  en la lista bibliográfica  final, y en ella  se le asignará
    asimismo una  etiqueta (como a  todas las obras  de la lista), pero  en el
    lugar del documento  en el que se encuentra el comando  \cmd nocite- no se
    imprimirá nada.

  \end{itemize}

  Ambos  comandos  pueden  recibir,  como  parámetro,  las  claves  de  varias
  referencias distintas, separadas  mediante comas.  Y en el  caso concreto de
  \cmd nocite-, si  se escribe \cmd nocite\{*\}- el efecto  será incluir en la
  lista bibliográfica final todas  las referencias bibliográficas incluidas en
  la base de datos.

\item[Qué  título tendrá  la lista  de referencias:]  Dependiendo del  tipo de
  documento  de que  se  trate, el  título  de la  lista  de referencias  está
  controlado  por dos  variables  de nombre  distinto\footnote{Lo  que, en  mi
    opinión,  es una  inconsistencia del  sistema. El  paquete  \ltr chbibref-
    incluye  un  comando  que  permite  cambiar  el  título  de  la  lista  de
    referencias con  independencia de que se  trate de un  documento tipo \ltr
    book- o  de un  documento tipo \ltr  article-.}.  En documentos  tipo \ltr
  book- para  cambiar el título de  la lista debe modificarse  el comando \cmd
  bibname-, mientras que en documentos tipo \ltr article- el comando \ltx- que
  hay que modificar es \cmd refname-.   El paquete \ltr babel-, por otra parte
  asigna a  cada uno de  estos comandos un  valor dependiente del  idioma.  En
  español los  nombres son ``Bibliografía'' para los  libros y ``Referencias''
  para los artículos.

  Si,  por   ejemplo,  deseamos  que   la  lista  de  referencias   se  titule
  ``Bibliografía citada'' y nuestro documento  es de tipo \ltr book- tendremos
  incluir  en el  cuerpo  de  nuestro documento\footnote{Y  es  que, según  mi
    experiencia, esta orden no produce  efectos si se incluye en el preámbulo.
    Posiblemente  por la  interacción que  la  carga del  paquete \ltr  babel-
    produce.} la siguiente orden:

  \verb|\renewcommand{\bibname}{Bibliograf\'ia citada}|

  Asimismo, si  se desea que  en el índice  de contenido de  nuestro documento
  aparezca una entrada para la bibliografía, hay que incluirla a mano mediante
  la orden \cmd addscontentline-  de \ltx-\footnote{El paquete \ltr tocbibind-
    hace que los índices y la lista de referencias se incluyan automáticamente
    en la tabla  de contenido.}, la cual debe  insertarse \emph{antes} de \cmd
  bibliography-,  ya que  si se  inserta detrás,  en el  índice  del documento
  aparecerá como página de la bibliografía la página en la que esta termina, y
  no la  página en  la que empieza\footnote{Véase  más adelante en  la sección
    \ref{sec:elementos-preamble}, a propósito de los elementos \ltr @preamble-
    de los ficheros \tpf bib-, cómo podemos incluir en un fichero \tpf bib- el
    código  necesario para  cambiar automáticamente  el título  a la  lista de
    referencias y asegurarnos de que se incluye en la tabla de contenido.}.
 
\end{description}

\subsection{Cómo generar la lista de referencias bibliográficas}
\label{sec:como-generar-la}

Una vez que nuestro documento  (llamémosle \ltr MiDoc.tex-) está preparado, en
los términos que se acaban de exponer, debemos seguir los siguientes pasos:

\begin{enumerate}[1º]

\item Compilar con \ltx- nuestro documento  \tpf tex-. Ello hará que se genere
  un fichero de  extensión \tpf aux-, en el que  se incluirá información sobre
  la base  de datos a  usar, el  fichero de estilo  a usar, y  las referencias
  bibliográficas que hay que incluir en la lista de referencias.

\item  Ejecutar, desde  la línea  de  comandos\footnote{Numerosas herramientas
    para \ltx-  incorporan la  posibilidad de ejecutar  desde el  mismo editor
    tanto \ltx- como  cualquiera de sus programas auxiliares,  en cuyo caso no
    sería preciso ejecutar \btx- \emph{desde  la línea de comando}.}, la orden
  \ltr   bibtex  MiDoc-.    Ello  hará   que   \btx-  lea   el  fichero   \tpf
  aux-\footnote{De hecho  \btx- se  ejecuta sobre el  fichero \tpf aux-,  y no
    sobre el fichero  \tpf tex-. Este detalle puede  escapársenos porque \btx-
    no exige que  se escriba la extensión. Pero si en  el comando incluimos la
    extensión  comprobaremos que  \ltr bibtex  MiDoc.tex- generaría  un error,
    mientras que \ltr  bibtex MiDoc.aux- funcionaría correctamente.}, generado
  por la anterior  compilación del documento, extrayendo de  él la información
  que necesita para  trabajar: qué base de datos debe usar,  qué estilo, y qué
  referencias hay que  buscar en la base.   Y así, tras extraer de  la base de
  datos  los registros  precisos,  y  formatearlos de  acuerdo  con el  estilo
  indicado,  \btx- genera  un fichero  de  extensión \tpf  bbl- en  el que  se
  contienen  los  comandos de  \ltx-  necesarios  para  escribir la  lista  de
  referencias bibliográficas  que hay que insertar en  el documento principal.
  \btx- genera también un fichero adicional,  de extensión \tpf blg- que es un
  fichero \tpf  log-\footnote{Es decir: en  él se almacenan todas  las salidas
    generadas por  \btx-, incluyendo los  mensajes de advertencia o  error. En
    caso de que algo no  haya funcionado como esperábamos es imprescindible la
    consulta de este fichero para ver qué es lo que ha podido fallar.}.

\item  Compilar de nuevo  el documento  principal con  \ltx-. En  esta segunda
  compilación, al leer  la orden \cmd bibliography-, se  insertará en su lugar
  el contenido del fichero \tpf bbl- generado en el paso anterior. Asimismo la
  nueva compilación reescribe el fichero \tpf aux-, añadiendo a la información
  que ya existía en él la generada ahora, que es más completa pues incluye los
  datos  de la  lista  bibliográfica final  que  se acaba  de  insertar en  el
  documento.

\item Esta última información es usada en una nueva compilación con \ltx- para
  escribir  correctamente los  rótulos que  hay que  colocar en  lugar  de los
  comandos  \cmd  cite-.   Y  eventualmente  puede  ser  necesaria  una  nueva
  compilación: cuando alguno de los campos  de la base de datos contenga algún
  comando de \ltx- que implique el uso de referencias cruzadas. En particular,
  el comando \cmd cite-.

\end{enumerate}

¿Parece complicado? No lo es. Lo  que ocurre es que lo he explicado incluyendo
detalles  sobre cómo  interactúan entre  sí los  distintos  ficheros generados
durante  las compilaciones  de \ltx-.   Simplificando  y sin  entrar en  tales
detalles,  lo que  hay  que hacer  es,  tras haber  indicado  en el  documento
principal el  nombre de la  base de datos  y del estilo, realizar  una primera
compilación con \ltx-, ejecutar \btx- y  luego compilar de nuevo con \ltx- dos
veces (o, en ciertos casos, tres).  Esta doble (o triple) compilación final no
es ninguna especialidad de \btx-, sino  que es requerida por \ltx- siempre que
trabaja con referencias cruzadas.  Asimismo  habrá que repetir el proceso cada
vez que incorporemos a nuestro  documento principal un comando que suponga una
modificación de lista bibliográfica final,  bien por cambiarle el estilo, bien
por incluir o eliminar referencias bibliográficas.

\section{Las bases de datos bibliográficas}
\label{sec:las-bases-de}

\subsection{Reglas generales sobre la escritura y codificación de los ficheros
  \tpf bib-}
\label{sec:ficheros-.bib-1}

Los ficheros \tpf bib- usados por  \btx- son ficheros de texto sometidos a las
siguientes reglas generales:


\begin{enumerate}

\item El tratamiento de los espacios  en blanco, tabuladores y saltos de línea
  es  similar  al  que  estos  caracteres reciben  en  \ltx-\footnote{Hay,  no
    obstante,  una diferencia,  y  es que  en  \ltx- una  línea  en blanco  es
    relevante pues sirve  para indicar un cambio de  párrafo.  En los ficheros
    \tpf bib-, una  línea en blanco es tan irrelevante como  un salto de línea
    simple.}. Son simples delimitadores de palabras y la regla es que da igual
  cuántos delimitadores haya entre dos palabras\footnote{Podríamos, en teoría,
    escribir todo el fichero \tpf bib- como una sola e inmensa línea en la que
    las palabras estuvieran separadas entre  sí por un solo espacio en blanco.
    Pero tal  fichero sería, para los  seres humanos, más difícil  de leer que
    otro en el  que hayamos usado los tabuladores, espacios  y saltos de línea
    para mejorar el aspecto visual (llamado \emph{legibilidad}) del fichero.}.

\item  \btx-  no  distingue   entre  mayúsculas  y  minúsculas\footnote{La  no
    distinción  entre mayúsculas  y minúsculas  se aplica  a los  elementos de
    \btx-  (nombres de  registros o  campos,  abreviaturas, etc),  pero no  al
    contenido de los campos de los registros.  }.

\end{enumerate}

\subsubsection*{Uso de caracteres españoles en los ficheros \tpf bib-}
\label{sec:uso-de-caracteres}

La  versión original  de \btx-  no  era capaz  de trabajar  con caracteres  no
anglosajones (como  eñes o vocales  acentuadas).  En la actualidad  existe una
versión de  \btx- que sí  puede hacerlo. No  obstante se suele  recomendar que
para  representar esos caracteres  que requieren  8 bits  se sigan  las reglas
generales  de  \ltx- y  así,  por  ejemplo, en  lugar  de  escribir \ltr  Pág-
escribamos ``\verb|P\'ag|''\footnote{El problema  de representar caracteres no
  anglosajones directamente en un fichero de texto, es que estos caracteres no
  están incluidos en la tabla  ASCII que todos los ordenadores reconocen, sino
  en  otras  tablas o  \emph{codificaciones}.   Y,  desgraciadamente para  los
  hablantes   de   idiomas   distintos   al  inglés,   existen   \emph{muchas}
  codificaciones distintas  para representar los  mismos caracteres: ``utf8'',
  ``latin1'', ``cp850'', etc.;  y para trabajar con un  documento que contenga
  este tipo de caracteres es preciso saber exactamente qué codificación se usó
  para almacenarlo  pero, desgraciadamente, los  ficheros de texto  no guardan
  información sobre su propia codificación. Además, en el caso concreto de los
  ficheros \tpf  bib-, el uso  de una codificación  u otra puede afectar  a la
  ordenación alfabética de los registros  almacenados en la base de datos.}. Y
aunque es  cierto que  escribir de  ese modo nuestro  idioma es  incómodo, hay
herramientas que lo  facilitan\footnote{Como, por ejemplo \ltr pybliographer-,
  una utilidad de  gestión de ficheros \tpf bib-  para sistemas Unix/Linux, en
  la que podemos teclear normalmente cualquier carácter, pero antes de guardar
  el fichero,  los caracteres no  anglosajones son convertidos a  la secuencia
  \ltx- necesaria para representarlos. De  esta herramienta se vuelve a hablar
  en la sección \ref{sec:herr-graf}.}.

En todo caso, si  decidimos introducir caracteres no anglosajones directamente
en el fichero \tpf bib-, debemos tener en cuenta que como el contenido de este
fichero puede eventualmente incorporarse a un documento \tpf tex-, si en ambos
el tipo de  codificación no coincidiera se generaría un  error. Por ello, para
usar estos caracteres deben cumplirse los siguientes requisitos:

\begin{enumerate}[a)]

\item Debemos  guardar el fichero \tpf  bib- con la misma  codificación con la
  que vayamos a guardar el fichero \tpf tex-  en el que se hará uso de la base
  de datos.

\item   En    el   fichero   \tpf    tex-   debe   usarse   la    orden   \cmd
  usepackage[cod]\{inputenc\}-    donde   ``cod''    significa    una   cadena
  representativa de  la codificación usada  en el fichero  \tpf bib- (y  en el
  \tpf tex-): \ltr utf8-, \ltr latin1-, etc.

\item Debemos asegurarnos de que estamos usando la versión de \btx- de 8 bits.

\end{enumerate}

\subsection{Registros y campos bibliográficos}
\label{sec:registros-y-campos}

Un fichero \tpf bib- almacena una base de datos bibliográfica. En toda base de
datos hay dos nociones fundamentales:

\begin{description}

\item[Campo:] Un dato  aislado y básico. Por ejemplo, el año  de edición de un
  libro, o  el nombre del  autor de  un artículo de  revista, o la  página del
  libro  de  las  actas  de  un  congreso en  donde  empieza  una  determinada
  ponencia...

\item[Registro:]  Un  conjunto  de  campos  que permiten  describir  de  forma
  completa un  ejemplar del  tipo de realidades  a que  se refiere la  base de
  datos. En nuestro caso, tratándose  de una base de datos bibliográfica, cada
  registro describe una referencia bibliográfica concreta (y completa).

\end{description}

Es  decir:  todos los  campos  que contienen  datos  de  una misma  referencia
bibliográfica,  constituirán un registro.   Y la  base de  datos, en  sí misma
considerada, no  será sino un  conjunto de registros,  cada uno de  los cuales
consta de varios campos.

\subsubsection{Formato general de los registros y los campos}
\label{sec:formato-general-de}

En los  ficheros \tpf bib-  se usa el  carácter ``@'', seguido de  una palabra
representativa del tipo de registro de  que se trate, para indicar que empieza
un registro. Asimismo se usan llaves para delimitar el contenido del registro.
En consecuencia el formato de un registro es el siguiente:


{\small
\begin{verbatim}
@TipoRegistro{clave,
   Campo1,
   Campo2,
   ...
   CampoN,
}
\end{verbatim}
}

Donde \emph{TipoRegistro}  es el nombre que  identifica a un  concreto tipo de
registro, y  puede escribirse indistintamente en mayúsculas  o minúsculas.  En
el contenido del registro, hay básicamente dos elementos:


\begin{description}

\item[La clave de identificación del  registro] es siempre su primer elemento.
  Se usa para distinguir a un  registro concreto del resto de los registros de
  la base de  datos, y por ello en  una misma base de datos no  debe haber dos
  registros que tengan  claves iguales, y tampoco deben  unirse o manejarse en
  un mismo documento dos o más bases de datos si ello implica que alguna clave
  vaya a repetirse\footnote{Digo que no \emph{deben} usarse claves iguales; no
    que no pueda  hacerse. Ello es porque \btx- no  genera ningún error cuando
    en un fichero  dos o más registros tienen la  misma clave. Simplemente, en
    tal caso, toma en consideración  exclusivamente la primera aparición de la
    clave, ignorando las restantes. Y,  a tales efectos, téngase en cuenta que
    para  el control  de  las claves  \btx-  no distingue  entre mayúsculas  y
    minúsculas.}.   La  clave  puede  consistir en  cualquier  combinación  de
  letras,  números y  ciertos  signos de  puntuación.   No se  admiten ni  los
  caracteres no anglosajones (vocales  acentuadas y los caracteres ``¡¿ñÑçÇ'')
  ni ciertos signos de puntuación como la coma (que se usa para indicar que la
  clave ha terminado).

  En las citas hechas en el documento \tpf tex- habrá que usar esta clave para
  identificar al registro al que se quiere hacer referencia.

\item[Los campos:]  A \btx-  le es  indiferente en qué  orden se  escriban los
  campos  de un determinado  registro, así  como si  se usan  cero, uno  o más
  saltos de línea entre dos campos.

\end{description}

Para  escribir un  campo dentro  de un  registro podemos  usar indistintamente
cualquiera de los dos formatos siguientes:

{\small
\begin{verbatim}
NombreCampo = { Contenido },
NombreCampo = " Contenido ",
\end{verbatim}
}

En ambos casos intervienen los siguientes elementos:

\begin{description}

\item[Nombre del  campo:] Normalmente será  uno de los campos,  obligatorios u
  opcionales,  previstos para  el tipo  de registro  de que  se trate.   Si el
  nombre no coincide con ninguno de los que el estilo ha previsto para el tipo
  de  registro que  sea, su  contenido será  ignorado.  Si  el mismo  campo se
  introduce  más  de  una vez  en  el  mismo  registro,  sólo será  tomada  en
  consideración su primera aparición.

\item[Delimitadores de  contenido:] Se exige  el uso de delimitadores  para el
  contenido del campo siempre que este  no tenga un valor puramente numérico o
  consista  en  una abreviatura  previamente  definida  como  tal mediante  la
  función  \ltr  @String-  (véase la  sección  \ref{sec:uso-de-abreviaturas}).
  Como  delimitadores se  pueden usar  las llaves  o las  comillas  dobles. La
  diferencia  entre usar  unas u  otras es  sólo una:  cuando  usamos comillas
  dobles como  delimitadores, dentro de  la cadena delimitada no  podemos usar
  comillas dobles, salvo que las encerremos entre llaves. Por ejemplo:

  {\small
\begin{verbatim}
title = {Comentarios a "El Buscón" de Quevedo},
title = "Comentarios a {"}El Buscón{"} de Quevedo",
\end{verbatim}
  }

  servirían para incluir un título que a su vez incluya dobles comillas.


\item[El contenido de  los campos:] cualquier cosa que se  incluya en un campo
  puede, eventualmente, incorporarse a un documento \ltx-, y por lo tanto:

  \begin{enumerate}

  \item Es  posible incluir,  como contenido de  un campo, comandos  de \ltx-,
    aunque ello sólo está recomendado en  casos muy especiales, y nunca es una
    buena idea que esos comandos sean de \emph{formato} de texto, pues la idea
    que está detrás  de \btx- es que el formateo de  las listas de referencias
    se haga depender del estilo usado.

  \item Dentro del  contenido del campo hay que  respetar las reglas generales
    de \ltx-, no  usar sus caracteres reservados (salvo cuando  se trate de un
    uso legal) y recordar que \ltx-, a diferencia de \btx-, sí distingue entre
    mayúsculas y minúsculas.

  \item Las  llaves que pueda haber  en un campo deben  estar equilibradas. Es
    decir: toda llave abierta debe ser  cerrada. Y ello se aplica incluso para
    llaves  que se  deben  imprimir\footnote{Lo que  se  debe a  que \btx-  no
      interpreta los comandos de \ltx-, por lo  tanto si en un campo se lee el
      texto \ltr  \textbackslash\{-, \btx- no ve un  caracter imprimible, sino
      una llave abierta que debe ser  cerrada dentro del campo. Por ello en el
      raro caso de que debamos incluir  como contenido de un campo el carácter
      ``\{`` (o  ``\}'') habría que usar  el comando \ltx-  \cmd leftbrace- (o
      \cmd rightbrace-).}.

  \end{enumerate}

  Existen  reglas especiales para  el contenido  de algunos  campos concretos,
  pero  estas  se  exponen  más   adelante,  cuando  se  explican  los  campos
  reconocidos por los estilos estándar de \btx-.

\item[La  coma  final:] Indica  que  el contenido  del  campo  ha terminado  y
  \emph{puede} empezar  otro campo  distinto. No es  obligatoria en  el último
  campo  de un  registro; pero  no se  produce ningún  error por  el  hecho de
  introducirla también en él.

\end{description}

Desde  el punto  de  vista de  un  concreto estilo  bibliográfico, los  campos
presentes en un registro se incluirán en uno de los siguientes grupos:

\begin{description}

\item[Campos  obligatorios:] Son  aquellos que  el estilo  en  cuestión espera
  encontrar  en un  determinado  tipo de  registro,  de manera  que  si en  un
  registro  concreto  alguno  de  ellos  no está  presente,  se  generará  una
  advertencia en  la salida estándar  y en el  fichero \tpf blg-  generado por
  \btx-. En algunas  ocasiones la falta de uno de  estos campos puede provocar
  un error, aunque eso no debería ocurrir en un estilo bien diseñado.

\item[Campos opcionales:] Son  aquellos para los que el  estilo en cuestión ha
  previsto  algún tipo  de  acción, pero  se  ha previsto  también su  posible
  ausencia, y esta no tiene especial trascendencia.

\item[Campos ignorados:] Cualquier  otro campo presente para el  que el estilo
  en cuestión no ha previsto ningún tipo de acción.

\end{description}


\subsubsection{Tipos de registros bibliográficos}
\label{sec:registr-bibl}

Las  referencias  bibliográficas  pueden  ser  de distinta  naturaleza,  y  en
consecuencia  se distinguen  diferentes  tipos de  registros.  A  continuación
expondré los  nombres de los distintos  tipos de registro  previstos en \btx-,
indicando  para  cada  uno  de  ellos sus  campos  obligatorios  y  opcionales
específicos\footnote{Con  \emph{específicos}  quiero  decir que,  para  evitar
  repeticiones, en la relación que sigue no  se ha incluido en cada uno de los
  registros los nombres de dos  campos opcionales que están presentes en todos
  los registros, sean del tipo que sean:  \ltr key- y \ltr note-.  }. Para que
\btx- pueda  reconocer adecuadamente  los registros y  campos, los  nombres de
estos deben  ser usados en  inglés, y por  ello los mencionaré en  ese idioma.
Entre  paréntesis añadiré, para  los tipos  de registro,  lo que  significa su
nombre.  En la próxima sección se explica para qué sirve cada campo.

\begin{description}

\item[Article  (artículo):]  Un artículo  publicado  en  una revista.   Campos
  obligatorios:    \texttt{author},    \texttt{title},   \texttt{journal}    y
  \texttt{year}.  Opcionales: \texttt{volume}, \texttt{number}, \texttt{pages}
  y \texttt{month}.

\item[Book   (libro):]   Un   libro   \emph{normal}.    Campos   obligatorios:
  \texttt{author}  o  \texttt{editor},  \texttt{title},  \texttt{publisher}  y
  \texttt{year}.      Opcionales:    \texttt{volume}     o    \texttt{number},
  \texttt{series}, \texttt{address}, \texttt{edi\-tion} y \texttt{month}.

\item[Booklet (folleto):] Un trabajo impreso y distribuido pero sin que conste
  la   editorial  o   institución  que   lo  patrocina.    Campo  obligatorio:
  \texttt{title}.  Campos  opcionales: \texttt{author}, \texttt{howpublished},
  \texttt{address}, \texttt{month} y \texttt{year}.

\item[Conference  (conferencia):]   Idéntico  a  InProceedings.    Se  incluye
  exclusivamente para mantener la compatibilidad con el formato \emph{Scribe}.

\item[InBook (dentro  de un libro):] Una parte  de un libro, que  puede ser un
  capítulo (o sección o similar) o  un rango de páginas, o ambas cosas. Campos
  obligatorios:    \texttt{author}    o    \texttt{editor},    \texttt{title},
  \texttt{chapter}  y/o  \texttt{pages},  \texttt{publisher} y  \texttt{year}.
  Opcionales:    \texttt{volume}     o    \texttt{number},    \texttt{series},
  \texttt{type},  \texttt{address},  \texttt{edition}  y  \texttt{month}.   El
  campo \texttt{title}, en estas referencias,  se refiere al título del libro,
  no al título del capítulo o grupo de páginas a que se refiere el registro.

\item[InCollection (en  una colección):]  Una parte de  un libro que  tiene su
  propio   título.   Campos  obligatorios:   \texttt{author},  \texttt{title},
  \texttt{booktitle},   \texttt{publisher}   y   \texttt{year}.    Opcionales:
  \texttt{crossref},   \texttt{editor},  \texttt{volume}   o  \texttt{number},
  \texttt{series},     \texttt{type},     \texttt{chapter},    \texttt{pages},
  \texttt{address}, \texttt{edition} y \texttt{month}.

\item[InProceedings (en  las actas):] Una conferencia, artículo  o ponencia en
  las actas  de un congreso o,  o, en general,  en un libro que  agrupe varios
  trabajos  de  autores  distintos   y  con  títulos  independientes.   Campos
  obligatorios:     \texttt{author},    \texttt{title},    \texttt{booktitle},
  \texttt{year}.    Campo   opcionales:  \texttt{crossref},   \texttt{editor},
  \texttt{volume}   o    \texttt{number},   \texttt{series},   \texttt{pages},
  \texttt{address},        \texttt{month},       \texttt{organization}       y
  \texttt{publisher}.

\item[Manual:]  Documentación  técnica.   Campo  obligatorio:  \texttt{title}.
  Campos opcionales: \texttt{author}, \texttt{organization}, \texttt{address},
  \texttt{edition}, \texttt{month} y \texttt{year}.

\item[MasterThesis (Proyecto  fin de carrera):]  Lo que en el  mundo académico
  norteamericano   se  denomina   ``\emph{Master  Thesis}''   o  ``\emph{Minor
    Thesis}''  y que  en España  equivale  a las  llamadas ``\emph{Tesinas  de
    licenciatura}'' y a  los \emph{Proyectos de fin de  carrera}; es decir: un
  trabajo de investigación menor,  leído en una institución académica.  Campos
  obligatorios:    \texttt{author},    \texttt{title},    \texttt{school}    y
  \texttt{year}.      Opcionales:     \texttt{type},    \texttt{address}     y
  \texttt{month}.

\item[Misc  (miscelánea):]  Este tipo  está  previsto  para  ser usado  cuando
  ninguno de  los otros tipos  encaje bien. Su  peculiaridad es que  carece de
  campos  obligatorios.  Campos  opcionales:  \texttt{author}, \texttt{title},
  \texttt{howpublished}, \texttt{month} y \texttt{year}.

\item[PhdThesis (tesis doctoral):] Lo que en el mundo académico norteamericano
  se denomina ``\emph{PhD Thesis}'',  ``\emph{Major Thesis}'' y que equivale a
  las  tesis  doctorales españolas,  es  decir:  un  trabajo de  investigación
  original y extenso que permite obtener el título de doctor que constituye la
  más   alta    graduación   académica   existente.     Campos   obligatorios:
  \texttt{author},    \texttt{title},    \texttt{school}   y    \texttt{year}.
  Opcionales: \texttt{type}, \texttt{address} y \texttt{month}.

\item[Proceedings (libro de actas):] El  libro de actas donde se encuentra una
  conferencia o ponencia en un congreso, aunque yo lo uso con carácter general
  para  todos aquellos  libros que  agrupan trabajos,  cada uno  de  ellos con
  distintos  autores.  Campos  obligatorios:  \texttt{title} y  \texttt{year}.
  Opcionales:    \texttt{booktitle},   \texttt{editor},    \texttt{volume}   o
  \texttt{number},    \texttt{series},    \texttt{address},    \texttt{month},
  \texttt{organization} y \texttt{publisher}.

\item[TechReport  (informe  técnico):]  Un  informe publicado  por  un  centro
  académico o  institución similar, normalmente numerado dentro  de una serie.
  Campos obligatorios: \texttt{author}, \texttt{title}, \texttt{institution} y
  \texttt{year}.     Campos   opcionales:    \texttt{type},   \texttt{number},
  \texttt{address} y \texttt{month}.

\item[Unpublished (no publicado):] Un documento que tiene un autor y un título
  pero   que  formalmente   no  ha   sido  publicado.    Campos  obligatorios:
  \texttt{author}, \texttt{title} y \texttt{note}.  Opcionales: \texttt{month}
  y \texttt{year}.

\end{description}

A los  campos mencionados hay que añadir,  en todos los tipos  de registro, un
campo \texttt{key}  y un campo  \texttt{note}, ambos opcionales, salvo  en las
referencias tipo \ltr Unpublished-, donde \texttt{note} es obligatorio.

\subsubsection{Significado de los distintos campos bibliográficos}
\label{sec:campos-de-datos-1}

\begin{description}

\item[Address:]  Este campo  está pensado  para almacenar  la dirección  de la
  editorial o institución responsable  de una publicación. Es siempre opcional
  y conviene usarlo,  sobre todo, cuando las editoriales  sean pequeñas o poco
  conocidas. En  la mayor parte  de los casos,  además, no se incluye  aquí la
  dirección completa, sino exclusivamente la ciudad donde reside la editorial.
  Hay  especialidades científicas  donde  es costumbre  identificar las  obras
  publicadas más bien  por la ciudad que por la editorial,  en cuyo caso puede
  ser buena idea indicar la ciudad no  en el campo \ltr address-, sino en \ltr
  publisher-.

\item[Author:] Como este es el  campo más complejo, dividiré su explicación en
  varios apartados:


  \paragraph{La autoría de las referencias:}
  \label{sec:la-autoria-de}

  Toda  referencia bibliográfica  ha  de  tener una  \emph{autoría}  y en  tal
  sentido,  en  todos  los tipos  de  registros  hay  algún campo  dirigido  a
  recogerla. Ese campo es normalmente \ltr author-, aunque:

  \begin{enumerate}[a)]

  \item  En \ltr  Proceedings- la  autoría viene  recogida por  el  campo \ltr
    editor-,  porque  este  tipo  de   registro  recoge  un  libro  sin  autor
    propiamente dicho,  que recopila  o agrupa varios  trabajos que  sí tienen
    autor.

  \item En  los tipos \ltr Book- e  \ltr InBook-, el campo  \ltr author- puede
    ser  sustituido  por  \ltr   editor-.   Si  se  especifican  ambos  campos
    simultáneamente, el contenido del campo \ltr editor- es ignorado.

  \item  En las  referencias  de tipo  \ltr  Manual-, a  falta  de campo  \ltr
    author-, la autoría se atribuirá al campo \ltr organization-.

  \end{enumerate}

  En cualquiera de estos casos, la recomendación es que el nombre del autor se
  escriba tan  completo como se  conozca, es decir:  que no se  usen iniciales
  para  sustituir  al nombre  propio,  ya  que  ese efecto  podemos  obtenerlo
  mediante el estilo  bibliográfico. Pero si en la base de  datos no consta el
  nombre completo no hay estilo bibliográfico que sea capaz de adivinarlo.

  \paragraph{Varios autores:}
  \label{sec:caso-en-el}

  Si  hay  más  de  un  autor,  para  separar  unos  de  otros  hay  que  usar
  necesariamente la palabra ``and'' para que \btx- sepa donde empieza un autor
  y donde  acaba otro.  Si  usamos la conjunción española  equivalente, ``y'',
  \btx- no formateará correctamente el contenido de este campo porque no sabrá
  separar a los distintos autores. El hecho de usar ``and'' dentro del fichero
  \tpf  bib- no  significa que  en nuestros  documentos se  vaya a  usar dicha
  palabra, pues eso depende del fichero de estilo.

  Si  \emph{dentro}  del nombre  de  uno de  los  autores  aparece la  palabra
  ``and'', hay  que encerrar entre  llaves (distintas de las  genéricas llaves
  para  delimitar el contenido  del campo)  el nombre  completo.  Así,  en los
  siguientes ejemplos:

  {\small
\begin{verbatim}
author = "{McArthur And Inc.}",
author = {{McArthur And Inc.}},
author = {Thomas R. Delan And {McArthur And Inc.}},
\end{verbatim}
  }

  El nombre  que contiene,  como parte  de él, la  palabra ``And''  va siempre
  entre llaves, con independencia de  qué tipo de delimitador se haya escogido
  para el campo y de si hay o no mas nombres.

  \paragraph{Distinción entre nombre y apellidos:}
  \label{sec:dist-entre-nombre}

  El nombre de una persona tiene  varias partes de las que las principales (en
  los nombres españoles) son el nombre propio (también llamado nombre de pila,
  o  nombre a  secas)  y los  apellidos.   \btx- descompone  los nombres  para
  determinar qué parte de los mismos hay que atribuir al nombre propio, cuál a
  los  apellidos, y  cual a  las  restantes partes  que \btx-  reconoce en  un
  nombre\footnote{En realidad  \btx- descompone los nombres  en cuatro partes:
    Nombre de  pila (en inglés,  \emph{First name)}, Partícula  de separación,
    Apellidos  (en  inglés  \emph{Last  name}  o  \emph{Surname}  y  partícula
    ``Jr.'',  muy habitual  en los  nombres anglosajones.   En las  líneas que
    siguen he omitido lo referente  a esta última partícula, inexistente fuera
    del  mundo anglosajón.   No  obstante si  en  nuestra base  de datos  debe
    introducirse algún nombre propio  que incluya dicha partícula, normalmente
    esta será reconocida por \btx- y su presencia no alterará las reglas que a
    continuación se exponen.}. Para ello se siguen las siguientes reglas:

  \begin{enumerate}[1ª.]

  \item Si  el texto introducido  consta de una  sola palabra, se  asignará al
    apellido.

  \item \label{Regla:2} Si consta de dos o más palabras, se considerará que la
    última es el apellido y el resto es el nombre de pila.  Esta regla se basa
    en  que fuera  de la  Península Ibérica  (y paises  de  tradición cultural
    ibérica), es usual tener un solo apellido y varios nombres de pila.

  \item En el  caso de que el  apellido conste de varias palabras  y alguna de
    las palabras  interiores estuviera escrita con  minúsculas, se considerará
    que tal palabra es una partícula  que separa el nombre del apellido por lo
    que se asignará al apellido todo lo que esté detrás de tal partícula, y al
    nombre  lo  que  esté  antes.   Y así,  por  ejemplo,  \btx-  interpretará
    correctamente el nombre

    {\small\verb|Miguel de Cervantes Saavedra|}

    siempre y cuando escribamos el ``de'' y lo hagamos en minúsculas.  Pero si
    escribimos:

    {\small
\begin{verbatim}
Miguel Cervantes Saavedra
Miguel De Cervantes Saavedra
miguel de cervantes saavedra
\end{verbatim}
    }

    como  \btx-   no  podrá  identificar   la  partícula  aplicará   la  regla
    \ref{Regla:2}, y  considerará que ``Saavedra''  es el apellido y  el resto
    nombre  propio\footnote{Por   ello  si  queremos  que   en  nuestra  lista
      bibliográfica las  partículas se impriman en mayúsculas,  pero que \btx-
      las  reconozca  correctamente,  habría  que  escribir:  ``\texttt{Miguel
        \{\textbackslash uppercase\{d\}e\} Cervantes Saavedra}''.}.

  \item Si solo conocemos el apellido,  y este consta de varias palabras, para
    evitar que  las primeras se  asignen al nombre  de pila, hay  que encerrar
    todo el nombre  entre llaves (además de las  delimitadoras del campo). Por
    ejemplo:

    {\small\verb|author = "{Ortega y Gasset}",|}

  \item Si en  un nombre se usa  una coma, \btx- asignará al  apellido todo lo
    que esté antes  que la coma, y  al nombre propio todo lo  que esté detrás.
    Esta regla prevalece sobre todas las anteriores.

  \end{enumerate}

  Para nombres  españoles, con dos apellidos,  lo más seguro es  usar una coma
  entre  el  apellido  y  el  nombre  para impedir  que  \btx-  realice  malas
  interpretaciones. Porque  aunque en ocasiones la partícula  nos puede ayudar
  (como en ``Miguel de Cervantes Saavedra'') la mayoría de los nombres carecen
  de partícula, y  en muchas ocasiones la partícula  produciría un error, como
  en ``Félix Lope de Vega y Carpio''.

\item[Booktitle:] Es el título de un libro, parte del cual está siendo citado.
  Se usa, por lo  tanto en las referencias que sirven para  citar una parte de
  un libro que  tenga un título distinto del título del  libro, es decir: \ltr
  InCollection- e \ltr InProceedings-. Este campo también está presente en las
  referencias de tipo \ltr Proceedings-, aunque en ellas se usa exclusivamente
  para las referencias cruzadas (véase la sección \ref{sec:refer-cruz-entre}).
  A este  campo le son  de aplicación las  reglas que más adelante  se exponen
  sobre el campo \ltr title-.

\item[Chapter:] Contiene  un número de capitulo, sección  o unidad estructural
  de un  libro. Se usa en las  referencias de tipo InBook  e InCollection para
  identificar una parte de un libro.

\item[Crossref:]  Contiene una  etiqueta  en  la base  de  datos para  generar
  referencias cruzadas internas (véase la sección \ref{sec:refer-cruz-entre}).

\item[Edition:] Se  trata del  número de  edición de un  libro. Es  siempre un
  campo opcional;  y de hecho sólo suele  rellenarse cuando no se  trata de la
  primera edición.  Los  estilos estándar presuponen que se  escribirá con una
  palabra en mayúsculas  (y en inglés).  Por ejemplo  ``Second'', aunque si se
  escribe  en español  (``Segunda''),  o  con un  ordinal  expresado en  forma
  numérica (``2ª''),  no pasa nada\footnote{En los estilos  estándar detrás de
    este campo se  añade la palabra ``edition'' y,  si están españolizados, la
    palabra ``edición'', por ello no  conviene usar directamente un número que
    no sea un  ordinal, ya que entonces la  referencia contendría la expresión
    ``2  edición'', que  no  es correcta.}.   Lo  que sí  es importante,  para
  generar listas bibliográficas consistentes, es que siempre lo escribamos del
  mismo modo.

\item[Editor:]  Este  campo  ya  ha  sido mencionado  a  propósito  del  campo
  ``author''.  El editor  de un libro es el que, sin  ser su autor propiamente
  dicho, lo impulsa y hace nacer.  Esta figura es importante hasta el punto de
  que sobre  él recae  la \emph{autoría} (a  efectos de  la base de  datos) en
  determinado tipo de  obras como puede ser una  antología, un estudio crítico
  de alguna obra  antigua, etc. Por ejemplo: una edición  crítica de la poesía
  barroca española, o el conocido ``\emph{Las mil mejores poesías de la lengua
    castellana}''.  Las reglas relativas a  este campo son similares a las del
  campo ``author''.

\item[Howpublished:] Este campo se usa  en las referencias de tipo ``Booklet''
  y ``Misc'' que recogen libros que no han sido publicados por una editorial o
  institución  ``normal''. En  este  campo  puede indicarse  cómo  se llegó  a
  publicar la obra de que se trate. La primera palabra debe ir en mayúsculas.

\item[Institution:] Este campo existe  exclusivamente en los informes técnicos
  (referencias \ltr TechReport-)  y recoge el nombre de  la institución que ha
  financiado el informe de que se trate.

\item[Journal:]  Para artículos publicados  en revistas  este campo  recoge el
  nombre  de la  revista.   En determinados  campos  del conocimiento  existen
  abreviaturas  estándar para  las revistas  más conocidas,  por lo  que puede
  usarse la abreviatura  en lugar del nombre completo de  la revista. Sobre el
  uso  de   abreviaturas  en   los  ficheros  \tpf   bib-  véase   la  sección
  \ref{sec:uso-de-abreviaturas}.

\item[Key:]  Está  disponible  como  campo  opcional en  todos  los  tipos  de
  referencias.  Se usa en relación con la generación de etiquetas, en aquellos
  estilos en los que estas se generan  a partir del nombre del autor, como por
  ejemplo ocurre en el estilo \ltr  alpha-. En tales casos, el valor del campo
  \ltr key- será  usado para la generación de la etiqueta  si no consta ningún
  valor para el campo \ltr author-,  incluso aunque si conste algún valor para
  alguno de  los campos que  en ciertos tipos  de referencias pueden  suplir a
  dicho campo (véase  lo que, al explicar el campo \ltr  author- se dijo sobre
  la autoría, en pág.  \pageref{sec:la-autoria-de}).

\item[Month:] En  artículos de revista se usa  para designar el mes  en el que
  salió el número que contiene dicho artículo. En otro tipo de trabajos el mes
  en el que  fueron publicados o terminados (para  trabajos no publicados). La
  documentación  oficial de  \btx- recomienda  introducir este  campo mediante
  unas abreviaturas estándar en inglés\footnote{\label{meses}Se trata de: \ltr
    jan-, \ltr  feb-, \ltr mar-, \ltr  apr-, \ltr may-, \ltr  jun-, \ltr jul-,
    \ltr aug-,  \ltr sep-, \ltr  oct-, \ltr nov-  y \ltr dec-.}, en  cuyo caso
  usando la abreviatura,  no habría que usar delimitadores  para el campo (por
  las  razones que  se explican  cuando  se habla  de las  abreviaturas en  la
  sección \ref{sec:uso-de-abreviaturas}).

  En teoría  aquí ocurre igual  que en el  campo \ltr author-, respecto  de la
  partícula ``and'' usada para separar entre  sí a los distintos autores, y es
  que aunque hayamos introducido el  mes en inglés (o mediante una abreviatura
  del nombre inglés),  luego, usando el estilo bibliográfico  correcto, en las
  referencias que  se inserten en  nuestros documentos aparecerá el  nombre en
  español.  Aunque, a  diferencia de lo que ocurre con  el campo \ltr author-,
  si aquí introducimos  la información directamente en español,  no se produce
  ningún error.

\item[Note:] Cualquier tipo de información adicional que no tenga cabida en el
  resto de  los campos\footnote{Yo lo he  usado, por ejemplo,  para recoger el
    nombre de los  traductores (en traducciones), o en  ediciones que han sido
    actualizadas por  personas distintas del  autor original, para  recoger el
    nombre de estos  últimos, o en documentación electrónica,  para recoger la
    dirección  de Internet  donde se  encuentra.  Aunque  para esto  último es
    preferible  usar un  campo adicional  de  nombre \ltr  url-, pues  existen
    varios estilos  no estándar  para \btx- que  preven la existencia  de este
    campo  y son  capaces de  formatearlo correctamente.   Incluso  el paquete
    makebst es  capaz de generar  automáticamente un estilo  bibliográfico que
    reconozca  este  campo.  Véase  la  sección  \ref{sec:makebst}.}.  En  los
  estilos bibliográficos estándar el contenido  de este campo se transcribe al
  final de la referencia y su primera palabra debería empezar en mayúsculas.

\item[Number:] Puede  tener distintos  significados. En los  informes técnicos
  recoge el número de informe de que  se trate. En artículos de revista se usa
  en  revistas   cuyos  distintos  ejemplares  se   identifiquan  mediante  un
  \emph{número}, o mediante  un número y un volumen. En libros  sólo se usa si
  el libro forma parte de una serie o colección. En los registros de tipo \ltr
  Book- y  \ltr Proceedings-  (así como en  sus derivados: \ltr  InBook-, \ltr
  InCollection-  e \ltr  InProceedings-) el  contenido de  este campo  sólo es
  tomado en consideración si el campo \ltr volume- se ha dejado en blanco.

\item[Organization:]  La  organización  que  financia una  conferencia  o  que
  publica un manual.  En los registros de tipo \ltr Manual- este campo designa
  la \emph{autoría} si \ltr author- carece de contenido.

\item[Pages:]  Se usa  en artículos  y en  partes de  libro para  designar las
  páginas concretas en  las que un trabajo se  encuentra. Los estilos estándar
  de \btx- esperan  que se pongan aquí las páginas  inicial y final, separadas
  por uno o dos guiones.  De  hecho estos estilos convierten el guión sencillo
  en  uno doble,  que es  el utilizado  en \ltx-  para designar  intervalos de
  páginas. Si se quiere indicar sólo la primera página, se utiliza el carácter
  ``+''  que significa  ``desde esa  página en  adelante''.  Como  los estilos
  estándar  suponen  que  se  trata  de   más  de  una  página,  en  la  lista
  bibliográfica final,  a veces al número  aquí puesto se le  añade la palabra
  ``páginas'' (\emph{pages}), o la abreviatura ``págs.'' (pp) en plural.

\item[Publisher:] La  editorial encargada de  la publicación de un  libro.  Su
  contenido suele complementarse con el del campo \ltr address-.

\item[School:]  Nombre de  la Escuela,  Facultad o  Instituto en  donde  se ha
  confeccionado   un   trabajo  académico   (tesis   doctoral   o  tesina   de
  licenciatura).

\item[Series:]  En libros  que  forman parte  de  una colección,  se usa  para
  recoger el nombre de la misma.

\item[Title:]  Es  un  campo  presente  en  todos  los  tipos  de  referencias
  bibliográficas, y es  siempre obligatorio (salvo en las  referencias de tipo
  \ltr Misc-).  Designa  el título del trabajo al  que nos estamos refiriendo,
  salvo  en las referencias  tipo \ltr  InBook-, donde  designa el  título del
  libro del que forma parte el capítulo  o rango de páginas al que nos estamos
  refiriendo.

  En teoría  los estilos  bibliográficos estándar \emph{normalizan}  (o pueden
  normalizar) el uso de las mayúsculas y minúsculas en los títulos. Y por ello
  en las referencias generadas a partir de nuestra base de datos no siempre se
  escribirán  los títulos  tal y  como  aparecen en  la base  de datos.   Para
  preservar  una letra  en mayúsculas  hay que  encerrarla entre  llaves.  Las
  llaves pueden  ponerse también alrededor de  una o varias  palabras. Esto es
  especialmente importante si  en el título se usan comandos  de \ltx-, ya que
  en ellos una  alteración de las mayúsculas y minúsculas  se traduciría en un
  error.

\item[Type:] Este campo existe en \ltr InBook-, \ltr InCollection-\footnote{La
    documentación oficial de \btx- y los  textos que he leído coinciden en que
    en los registros  \ltr InCollection- el campo \ltr  type- es opcional. Las
    pruebas que  yo he hecho  me llevan a  pensar que en estos  registros este
    campo  es  ignorado.   },  \ltr  MasterThesis-,  \ltr  PhdThesis-  y  \ltr
  TechReport-.   En los tres  últimos tipos  mencionados los  estilos estándar
  añaden a la referencia, como  texto aclaratorio, el nombre del registro. Por
  ejemplo: en una tesis doctoral, se escribe: ``\emph{Tesis doctoral}''. Si en
  el  campo \ltr  type- hay  algún texto,  se usará  este texto  en  lugar del
  predeterminado.  En  los registros  \ltr InBook-, aunque  por defecto  no se
  escribe ningún texto, sí  se escribirá lo que se diga en  este campo. Por lo
  tanto  aquí puede  escribirse una  pequeña aclaración  respecto del  tipo de
  trabajo de que se trata, como: ``Conferencia'', ``Disertación'', etc.

\item[Volume:] En artículos  de revista se usa para indicar en  qué tomo de la
  revista se ha  publicado dicho artículo.  En libros que  formen parte de una
  serie, para  indicar el tomo o volumen,  dentro de la serie  en cuestión. En
  este último  caso si  se da algún  valor al  campo \ltr volume-  los estilos
  estándar ignorarán el contenido del campo \ltr number-.

\item[Year:] El año de la edición  del libro o de publicación del artículo. Es
  otro de los  datos presentes (de forma obligatoria u  opcional) en todos los
  tipos  de  registros.   Hay  incluso  estilos bibliográficos  en  donde  las
  referencias se ordenan por  el valor de este campo, y las  citas en el texto
  lo incluyen también.

\end{description}

\subsubsection{Algunos consejos sobre tipos de registros y campos}
\label{sec:algun-cons-sobre}

Cuando  empezamos a  manejar  \btx- la  gran  cantidad de  tipos de  registros
existentes (catorce  en total) puede abrumar.  A  continuación expongo algunos
consejos, dictados por la experiencia, sobre cómo debemos proceder.

\begin{enumerate}

\item  Hay  tres  tipos  de  referencia básica:  artículos  de  revista  (\ltr
  article-), libros completos (\ltr book-) y partes de libro.  Para las partes
  de libro a su  vez hay tres posibilidades: Si la parte  se identifica por un
  número de sección  o de páginas, la referencia correcta  es \ltr InBook-, si
  se identifica  por un título,  yo distingo, a  su vez, según cada  parte del
  libro tenga un  autor distinto, en cuyo caso  utilizo \ltr InProceedings-, o
  todo el  libro pertenezca al  mismo autor, en  cuyo caso prefiero  usar \ltr
  InCollection-. Ello  es porque no me  tomo demasiado en serio  el nombre del
  tipo de referencia, y atiendo más bien al contenido de los campos\footnote{Y
    no hay que escandilarse por ello.   El propio autor de \btx- recomienda en
    \cite{patashnik88}  no  tomarse  excesivamente  en serio  los  nombres  de
    registros y campos.  }.

\item Tampoco hay  que tomarse demasiado en serio el nombre  de los campos. En
  ocasiones puede ser  conveniente, por ejemplo, incluir la  ciudad de edición
  junto con  el campo  reservado a la  editorial, y  en otros casos  puede ser
  preferible  usar para ello  el campo  \ltr address-.   Lo importante  es que
  seamos consistentes en nuestras decisiones.

\item La distinción entre campos  obligatorios y opcionales es importante sólo
  para  confirmarnos que  hemos elegido  el tipo  de registro  adecuado.  Pero
  tampoco es una tragedia que en  un caso concreto, algún campo obligatorio se
  quede sin rellenar.

\item Para la información que consideremos  relevante pero para la que no haya
  ningún campo que  la haya previsto, podemos usar el  campo \ltr note-.  Pero
  también podemos usar  campos específicos creados por nosotros.   En tal caso
  los estilos  estándar de \btx- la  ignorarán, pero podremos  crear un estilo
  bibliográfico propio que la tenga en cuenta. Yo, por ejemplo, utilizo muchas
  traducciones, y  estimo que  el nombre del  traductor es un  dato importante
  para la  referencia.  Empecé usando para  ello el campo \ltr  note-, pero al
  final diseñé un campo opcional  llamado \ltr traductor-.  Sobre la escritura
  de estilos bibliográficos propios, véase  la última parte de este documento.
  Aunque cuando  decidamos crear un campo  adicional, si el tipo  de datos que
  queremos incluir en él es relativamente corriente, puede merecer la pena que
  antes de bautizar  el campo, nos demos un paseo por  Internet; porque es muy
  posible que  alguien haya  ya sentido la  necesidad de recoger  tales datos,
  haya creado  ese campo y, lo que  es más importante, haya  escrito un estilo
  bibliográfico que sepa qué hacer con él.

\item  También   es  interesante  la  creación  de   campos  adicionales  para
  información que  es importante para  nosotros tener recopilada, pero  que no
  importa a los  lectores de nuestros documentos y no  tiene por qué incluirse
  en las  referencias bibliográficas  generadas. Por ejemplo  yo uso  para los
  libros un campo  llamado \ltr tejuelo- en el que  almaceno la ``etiqueta'' o
  ``tejuelo'' asignada  a determinado libro en  la biblioteca de  mi centro de
  trabajo, lo que me ayuda a localizar físicamente el libro con rapidez.

\item Por último, aunque he señalado que no hay que tomarse demasiado en serio
  ni  los nombres  de los  tipos de  registro,  ni los  de los  campos, ni  su
  naturaleza obligatoria  u opcional, y  he añadido que puede  ser interesante
  crear campos propios,  todo ello no debe llevarnos a  diseñar y escribir una
  base de  datos que sólo nosotros  seamos capaces de entender  y utilizar. El
  sentido  último de  las herramientas  de \ltx-  (y en  general  del software
  libre) es el de \emph{compartir} el trabajo:  Que lo que ya ha sido hecho no
  tenga por qué repetirse  (si está bien hecho y el autor  autoriza su uso por
  otros).  Cuando empezamos  a escribir nuestra base de datos  no hay forma de
  saber si terminará siendo tan importante que merezca la pena compartirla con
  otros y por  ello es mejor empezar  a escribirla con la mente  puesta en que
  tal vez en el futuro debamos compartirla.

\end{enumerate}

\subsection{Otras cuestiones relativas a los ficheros \tpf bib-}
\label{sec:otras-cuest-relat}

\subsubsection{Uso de abreviaturas y concatenación de cadenas}
\label{sec:uso-de-abreviaturas}

En un  fichero \tpf bib- es  posible definir abreviaturas  que simplifiquen la
escritura  de los  registros y  eviten  errores.  Para  usar una  abreviatura,
primero hay que definirla de acuerdo con el siguiente formato:
 
\verb|@String{Abreviatura = "Texto sin abreviar"}|

El nombre  de la abreviatura  debe empezar por  una letra y no  puede contener
caracteres en blanco ni ninguno de los siguientes caracteres:

\verb|" # % ' ( ) , = { }|

El texto  sin abreviar  se debe especificar  entre comillas.  Las  llaves para
delimitar el contenido de ``\texttt{@String}'' son obligatorias.

La definición se  puede hacer en cualquier lugar del fichero  \tpf bib- que no
sea el interior de un registro (o de algún otro elemento admitido en este tipo
de ficheros). El único  requisito en este punto es el de  que la definición de
una  abreviatura debe  hacerse \emph{antes}  de que  ésta sea  usada  en algún
campo.

Las abreviaturas  no pueden usarse  para los nombres  de campo o  de registro,
sino  exclusivamente en el  contenido del  campo. Y  como las  abreviaturas no
pueden estar encerradas  dentro de los delimitadores de  campo, cuando todo el
contenido del campo  es una abreviatura, ese campo  no requerirá delimitadores
de contenido. Así en el siguiente ejemplo:

{\small
\begin{verbatim}
@STRING{TeX = "The {T}e{X} {B}ook"}

@Book{eijkhout,
   author = {Victor Eijkhout},
   title = {TeX by topic},
   publisher = {Addison Wesley},
   year = 1992,
}

@Book{knuth,
   author = {Donald E. Knuth},
   title = TeX,
   publisher = {Addison Wesley},
   year = 1986,
}
\end{verbatim}
}

\noindent Hemos definido la abreviatura  ``TeX'' y dos registros, en el título
del primero de ellos aparece la cadena ``TeX'', pero \btx- no la toma como una
abreviatura,  porque  tal  cadena  se  encuentra dentro  de  unas  llaves  que
delimitan el contenido de un campo.  Por el contrario, en el segundo registro,
el texto ``TeX'' usado como contenido  del campo título, es tomado como un uso
de la abreviatura, dado que no  está dentro de ningún tipo de delimitadores de
contenido  de campo\footnote{En  el  ejemplo he  escrito  para la  abreviatura
  ``TeX'',  pero,  como \btx-  no  distingue  entre  mayúsculas y  minúsculas,
  habríamos obtenido el mismo resultado escribiendo ``TEX'' o ``tex'' tanto en
  la  definición  de la  abreviatura  como más  tarde  cuando  ésta es  usada.
  Podríamos también haber definido ``TEX'' y luego usar ``tex'' o ``TeX''.  }.

Si  la abreviatura  no cubre  todo el  contenido del  campo,  entonces debemos
escribir  entre  delimitadores  normales  el  contenido  no  cubierto  por  la
abreviatura, y \emph{concatenar} dicho contenido con el de la abreviatura.

En los  ficheros \tpf  bib- el  carácter para concatenar  cadenas de  texto es
``\#''.  Este  carácter permite concatenar  abreviaturas entre sí,  cadenas de
texto entre  sí y abreviaturas  con cadenas de  texto.  Por ejemplo, si  en el
fichero  \tpf bib-  de nuestro  anterior ejemplo  tuviéramos un  libro llamado
``\emph{Comentarios a `The TeX Book'  de Knuth}'', y quisiéramos aprovechar la
abreviatura  que  hemos  definido  para  ``\emph{The  TeX  Book}'',  podríamos
escribir el campo \ltr title- de dicho libro de la siguiente manera:

\verb|title = {Comentarios a `} # TeX # {' de Knuth},|

Hay tipos de  datos, como el nombre de las revistas  científicas, para los que
es  muy corriente  el uso  de  abreviaturas.  De  hecho en  ciertos campos  de
conocimiento  circulan  ficheros  en   el  que  se  recogen  las  abreviaturas
corrientemente  usadas para  las  revistas más  importantes.   En tales  casos
podemos usar dichos ficheros simplemente  indicando su nombre en la orden \cmd
bibliography-  antes del  nombre de  nuestra  base de  datos bibliográfica,  y
separando ambos nombre por una coma.

La siguiente línea, por ejemplo, provocaría que ``\texttt{MiBiblio.bib}'' (que
cabe  suponer que  es nuestra  base de  datos) se  concatenara con  el fichero
distribuido  por  la \emph{American  Mathematical  Society}  que contiene  las
abreviaturas habitualmente usadas para las revistas matemáticas:

\verb|\bibliography{mrabbrev,MiBiblio}|

En  los estilos estándar  de \btx-  hay ya  varias abreviaturas  definidas por
defecto: Están definidos los nombres de los meses del año en inglés (véase las
abreviaturas  que  mencioné  al  hablar  del  campo \ltr  month-  en  la  nota
\ref{meses}), así como  el nombre de varias revistas dedicadas  al campo de la
informática\footnote{Las  abreviaturas definidas  en el  estilo plain  son las
  siguientes:  acmcs (ACM  Computing Surveys),  acta (Acta  Informatica), cacm
  (Communications  of   the  ACM),  ibmjrd   (IBM  Journal  of   Research  and
  Development),  ibmsj (IBM  Systems  Journal), ieeese  (IEEE Transactions  on
  Software  Engineering), ieeetc  (IEEE Transactions  on  Computers), ieeetcad
  (IEEE  Transactions on  Computer-Aided Design  of Integrated  Circuits), ipl
  (Information Processing  Letters), jacm (Journal of the  ACM), jcss (Journal
  of  Computer and System  Sciences), scp  (Science of  Computer Programming),
  sicomp  (SIAM Journal  on  Computing), tocs  (ACM  Transactions on  Computer
  Systems), tods (ACM Transactions on Database Systems), tog (ACM Transactions
  on Graphics),  toms (ACM Transactions on Mathematical  Software), toois (ACM
  Transactions  on Office  Information Systems),  toplas (ACM  Transactions on
  Programming Languages and Systems), tcs (Theoretical Computer Science)}.

\subsubsection{Referencias cruzadas entre registros}
\label{sec:refer-cruz-entre}

En los ficheros \tpf bib- se admiten dos tipos de referencias cruzadas:

\begin{enumerate}

\item Usando el comando de \ltx- \cmd cite-.

\item Mediante el campo \ltr crossref-.

\end{enumerate}

En cuanto al uso  de \cmd cite-, podemos usarlo dentro de  un campo para hacer
referencia a otro  registro. Normalmente esto se hace en  el campo \ltr note-,
pero se puede hacer en cualquier otro.  Esto provocará que al citar en nuestro
documento \ltx- una  referencia que a su vez cita a  otra, ambas se consideren
citadas en  el documento,  y se incluyan  las dos  en la lista  de referencias
bibliográficas\footnote{Este es  uno de los  casos en los que  \btx- requerirá
  una compilación  adicional. Porque  el comando \cmd  cite- incorporado  a un
  registro bibliográfico no será compilado  por \ltx- hasta que se compila una
  versión  que incorpore el  contenido del  fichero \tpf  bbl-.  E  incluso es
  posible  que al incorporar  esta nueva  cita, sea  preciso, tras  la segunda
  ejecución de \ltx-, ejecutar de nuevo \btx-.}.

Pero también  es posible  usar referencias cruzadas  de otra manera:  para que
\btx- sepa  que el valor  de ciertos campos  que normalmente forman  parte del
contenido de un determinado tipo de registro, pero que en un registro concreto
no  han  sido incluidos,  debe  ser tomado  de  otro  registro distinto.   Por
ejemplo, si tenemos un libro homenaje a cierto autor, en el que varios autores
han  colaborado, cada uno  de ellos  con un  artículo.  Si  quisiéramos volcar
todos estos artículos a nuestra base de datos, podríamos:

\begin{enumerate}

\item Crear  tantos registros  como artículos  haya e incluir  en cada  uno de
  ellos todos los  datos requeridos, gran parte de  los cuales serán idénticos
  en todos los registros.

\item  Abreviar  mediante  referencias  cruzadas,  para  los  datos  idénticos
  escribirlos una sola vez.

\end{enumerate}

Para seguir el segundo procedimiento, deberíamos empezar por crear un registro
del  tipo ``Proceedings''  para el  libro  homenaje en  sí mismo  considerado,
rellenar  en él  el valor  de todos  los  campos que  podamos, y  luego en  el
registro correspondiente a cada uno  de los artículos extraidos de dicho libro
(que serían referencias de tipo  \ltr InProceedings-), usar como valor para el
campo \ltr  crossref-, el de  la clave que  hayamos asignado al  registro \ltr
Proceedings-\footnote{De  hecho, si  lo pensamos  bien, el  único  sentido que
  tiene la presencia  del campo \ltr booktitle- en los  registros de tipo \ltr
  Proceedings- es el de  que este campo se use para llenar  el campo del mismo
  nombre en los registros \ltr InProceedings- que apunten a este.}.

El efecto de todo ello será que:

\begin{enumerate}

\item En los registros \ltr InProceedings- que escribamos para cada uno de los
  artículos  contenidos en el  libro, podremos  no incluir  los campos  que ya
  están en el registro correspondiente  al libro general, pues al apuntar \ltr
  Crossref- al registro  correcto, \btx- sabrá que los datos  que faltan en el
  primer registro  deben ser tomados del  segundo. Esto, además  de reducir el
  tamaño de  la base de  datos, y ahorrar  tiempo para la introducción  de los
  mismos,  contribuye  a  garantizar  que  todas las  veces  que  estos  datos
  aparezcan, estén escritos exactamente igual.

\item Si en un documento \ltx- citamos al menos dos artículos que apunten a un
  mismo registro, automáticamente se incluirá en la lista final de referencias
  bibliográficas, además de los trabajos directamente citados por nosotros, el
  registro correspondiente al libro que los contiene.

\end{enumerate}

Para  que lo  anterior  funcione  correctamente, es  preciso  que el  registro
\emph{referenciado}  mediante  el  campo  \ltr  crossref-  (el  registro  \ltr
Proceedings-) se encuentre en la base de datos \emph{después} de los registros
referenciadores (los registros \ltr InProceedings-).

Los estilos estándar  de \btx- admiten referencias cruzadas  en los siguientes
casos:

\begin{enumerate}

\item  Los registros  de tipo  \ltr InProceedings-  pueden hacer  referencia a
  registros \ltr Proceedings- o \ltr Book-.

\item  Los registros  de tipo  \ltr  InCollection- pueden  hacer referencia  a
  registros \ltr Book-.

\end{enumerate}

\subsection{Herramientas para manejar ficheros \tpf bib-}
\label{sec:herr-para-manej}

Existen varias herramientas para la creación y mantenimiento de bases de datos
de  \btx- cuya  utilidad fundamental  es  la de  asegurarse de  que todos  los
registros están correctamente  escritos y de que no  hemos cometido errores ni
en el nombre del tipo del registro ni en el de los campos.

Como  yo  trabajo  exclusivamente  en  sistemas Linux,  en  este  apartado  me
concentraré en  herramientas que funcionen  en tales sistemas.   Para sistemas
MS-DOS, Mac-OS  o Microsoft  Windows existen otras  herramientas que yo  no he
probado y  sobre las que, en  consecuencia, no puedo  opinar. Existen asimismo
herramientas  escritas  en  java   que,  en  teoría,  funcionan  en  cualquier
plataforma, aunque yo no las he probado. He leído buenos comentarios sobre dos
de ellas: ``jbibtexmanager''  y ``jabref'', ambas son fáciles  de localizar en
Internet.

\subsubsection{Herramientas gráficas}
\label{sec:herr-graf}


Existen varias  herramientas gráficas que permiten trabajar  con ficheros \tpf
bib- mediante  ventanas, botones, barras de herramientas,  etc. Las utilidades
ofrecidas  por estas  herramientas  son bastante  similares.   Entre ellas  he
probado (porque están disponibles directamente en Debian):

\begin{description}

\item[pybliographer:]  Posiblemente  sea la  más  completa  de  todas las  que
  conozco (aunque admito que no es la que más me gusta, pues es muy difícil de
  manejar  sin  el  ratón, y  yo  tengo  alergia  al  ratón). Frente  a  otras
  herramientas ofrece las siguientes ventajas: que funciona también en modo de
  consola, que además del formato  \btx- puede manejar otros formatos de bases
  de datos bibliográficas  y, lo que es más importante:  es capaz de convertir
  bases de datos desde unos formatos a otros.

\item[Gbib:] Es  mi herramienta favorita  en este grupo\footnote{Lo que  no es
    mucho decir,  pues yo soy  usuario de  GNU Emacs (del  que me ocupo  en la
    próxima  sección).}.  Forma  parte  del proyecto  Gnome,  aunque no  suele
  instalarse por defecto.  Posiblemente  su principal inconveniente esté en el
  hecho de que no permite controlar el  tipo de codificación en el que hay que
  grabar  el  fichero,  sino  que  éste  es  guardado  automáticamente  en  la
  codificación  preestablecida para  el sistema\footnote{Esto  puede ocasionar
    una tediosa labor  de \emph{reparación}. Sobre todo si  se tiene en cuenta
    que hasta hace poco la codificación estándar para sistemas Linux españoles
    era  \ltr  latin1-   o  \ltr  latin15-  y  en   los  últimos  tiempos  las
    distribuciones  empiezan a  migrar a  \ltr utf8-.   De modo  que si  en un
    sistema  instalado recientemente abrimos  una base  de datos  escrita hace
    algún  tiempo, introducimos  cambios  en  ella y  los  guardamos, cabe  la
    posibilidad de que, si no tenemos control sobre las codificaciones usadas,
    todos  los  caracteres  no  anglosajones  del fichero  se  transformen  en
    caracteres ininteligibles que luego habrá que restuarar a mano.  }, aunque
  en realidad este  defecto es predicable de las  tres herramientas que recojo
  en la presente sección.

\item[Tkbibtex:] Una herramienta muy  parecida a las anteriores: permite crear
  bases  de datos  nuevas,  abrir bases  existentes,  añadir registros,  hacer
  búsquedas, etc. Apenas le he echado un vistazo.

\end{description}

\subsubsection{El modo \btx- en un editor de texto}
\label{sec:el-modo-bibtex}

Dado que  los ficheros \tpf  bib- son ficheros  de texto, otra  posibilidad es
editarlos con algún editor de texto  que esté provisto de alguna extensión que
facilite de alguna manera el trabajo con ellos.

\subsubsection{Otras herramientas disponibles en sistemas linux}
\label{sec:otras-herr-disp}

Además de las indicadas  existen muchísimas otras herramientas disponibles.


\section{Los estilos bibliográficos}
\label{sec:los-estil-bibl}

\subsection{Descripción de los estilos estándar de \btx-}
\label{sec:los-estilos-estandar-1}

\btx- se  acompaña de cuatro estilos  llamados \ltr plain-,  \ltr abbrv-, \ltr
alpha- y \ltr unsrt-.  El estilo esencial es el primero, y los otros funcionan
como él en todo salvo en uno o dos puntos concretos.

A  continuación  describiré  estos  estilos atendiendo  exclusivamente  a  sus
características  principales   y  sin  entrar  en   excesivos  detalles  (cuya
descripción  con palabras,  por otra  parte, ocuparía  demasiado  espacio). Mi
consejo para ver exactamente las características de cada uno de los estilos es
escribir un  fichero \tpf bib- de prueba,  incluir un registro de  cada uno de
los tipos admitidos, llenar \emph{todos} los campos posibles de cada registro,
e incluir  dicha base de datos  en un documento  de prueba, en el  que hayamos
incluido un comando \cmd nocite\{*\}-.

En  la próxima explicación,  menciono varias  veces la  noción \emph{autoría}.
Con  ella   me  quiero  referir  a  lo   que  ya  se  explicó   en  la  página
\pageref{sec:la-autoria-de}.

\begin{description}

\item[Estilo ``plain''.] Sus características son:

  \begin{enumerate}

  \item La lista bibliográfica final se ordena alfabéticamente atendiendo a la
    autoría, y si hubiera  más de una obra del mismo autor,  se toma en cuenta
    al año de las mismas y  después el título. Si sigue habiendo igualdad tras
    aplicar los criterios anteriores, el último criterio es el del orden en el
    que  fueron citadas y,  para obras  citadas simultáneamente  mediante \cmd
    nocite\{*\}-, el orden que tengan en la base de datos.

  \item Las  obras incluidas en la  lista son numeradas  consecutivamente y el
    número asignado a  cada una de ellas, entre corchetes,  se convierte en la
    etiqueta identificativa de la misma que será impresa en el lugar en el que
    se  encuentren  los  comandos  \cmd  cite- existentes  en  el  cuerpo  del
    documento.

  \item Los datos de los campos se incluyen completos.

  \item Para ciertos campos se  añaden determinadas palabras o abreviaturas en
    inglés.  Por ejemplo,  tras el contenido del campo  \ltr edition- se añade
    la  palabra ``\emph{edition}'',  y hay  tipos de  registro en  los  que el
    nombre del campo (en inglés) forma parte de la referencia.

  \end{enumerate}

\item[El  estilo ``abbrv'':] Es  idéntico al  estilo \ltr  plain- salvo  en el
  hecho de que para ciertos datos se usan abreviaturas y así el nombre de pila
  de  los autores  es sustituido  por  sus iniciales  y el  nombre de  ciertas
  revistas  (que están  predefinidas  en el  estilo  y que  se  refieren a  la
  informática) es sustituido  por su abreviatura.  Para la  mayor parte de los
  usuarios,  que no  usan las  revistas predefinidas  en el  estilo,  la única
  diferencia con \ltr  plain- es que del nombre propio del  autor sólo se usan
  las iniciales.

\item[El estilo ``alpha'':] Se distingue del estilo \ltr plain- exclusivamente
  en el hecho de que la etiqueta de identificación de cada obra en la lista no
  es  un  número,  sino un  texto  generado  automáticamente  a partir  de  la
  \emph{autoría} de la  referencia, el año de publicación  y, en ocasiones, el
  inicio del título\footnote{Como regla se  toman las tres primeras letras del
    nombre del autor y las dos últimas cifras del año de publicación. En obras
    con varios autores  se toma la inicial de los tres  primeros seguida de un
    signo ``+'' en formato de superíndice,  si hay más de tres autores. Si con
    estos criterios a  dos referencias se les asignara  la misma etiqueta, las
    obras se  ordenan relativamente  entre sí, atendiendo  a los  criterios de
    ordenación del  estilo \ltr plain-, y  a las etiquetas se  añade una letra
    para  diferenciarlas entre  sí: una  ``a'' a  la primera,  una ``b''  a la
    segunda, etc.}.  Asimismo la lista bibliográfica se ordena alfabéticamente
  según las etiquetas asignadas, y el  comando \cmd cite- imprime, en el lugar
  en el que se encuentre, la etiqueta asignada a la obra citada.

\item[El estilo ``unsrt'':]  Es igual al estilo \ltr plain-  salvo en el hecho
  de que en él la lista bibliográfica no se ordena alfabéticamente, sino según
  el orden en  el que las distintas obras que aparecen  en ella fueron citadas
  por  primera vez. En  este caso,  para las  obras incluidas  en la  lista de
  referencias  mediante un  comando  \cmd nocite-  se  considerará que  fueron
  citadas en el  lugar en el que  se encuentre tal comando. Y  para el comando
  \cmd nocite\{*\}-, se usará el orden en el que las referencias se encuentren
  en el fichero \tpf bib-, pero sólo para las referencias que no hubieran sido
  citadas antes de \cmd nocite\{*\}-.

\end{description}

En  cuanto a  las  características concretas  de  formateo de  los campos  son
similares en todos estos estilos. Los aspectos más llamativos son:

\begin{enumerate}

\item Se escribe  el nombre de pila antes  que el apellido, lo que  no deja de
  ser chocante en aquellos casos en  los que la lista de referencias se ordena
  alfabéticamente  a partir  de  los  apellidos (estilos  \ltr  plain- y  \ltr
  abbrv-,  ya  que  ello   provoca  una  ordenación  alfabética  tomando  como
  referencia, no  la primera  palabra de  cada párrafo, sino  la segunda  o la
  tercera.

\item Los distintos  elementos de la referencia se  separan mediante un punto.
  Lo  que también  provoca problemas  cuando el  contenido de  un campo  no es
  numérico y  no empieza con una  mayúscula, pues los estilos  sólo ajustan el
  uso de las mayúsculas en el campo \ltr title-.

\item  Todos los  campos aparecen  con letra  normal, salvo  el título  de los
  libros y  trabajos académicos y el  nombre de las revistas,  que aparecen en
  cursiva.

\end{enumerate}

\subsection{Estilos adicionales}
\label{sec:estilos-adicionales}

Junto con estos estilos estándar,  existen multitud de estilos disponibles. En
la  CTAN  hay  varios  de   ellos  preparados  para  su  descarga,  y  algunas
distribuciones    de   \ltx-    incluyen   por    defecto    ciertos   estilos
adicionales\footnote{Por ello,  antes de buscar  estilos adicionales, conviene
  que  nos  aseguremos  de  cuáles   son  los  estilos  incluidos  en  nuestra
  distribución de  \ltx-. Lo que  se puede hacer por  distintos procedimientos
  que  dependen  de  nuestra  concreta  instalación. A  falta  de  algún  otro
  procedimiento específico, podemos buscar los ficheros de extensión \tpf bst-
  existentes  en nuestro sistema;  o mejor:  los existentes  en alguno  de los
  directorios  en los  que \btx-  busca los  ficheros de  estilo.  En  la nota
  \ref{nota:cnf} se explica  cómo localizar cuáles son esos  directorios en un
  sistema Unix/Linux.}.

Entre los  estilos adicionales hay  algunos cuya instalación es  tan corriente
que  son estilos  \emph{casi}  estándar.   Otros no  son  tan conocidos,  pero
resultan  interesantes.  Muchos  de ellos  consisten exclusivamente  en  uno o
varios estilos para \btx-, y  otros incluyen también un \emph{paquete} que hay
que incluir en  el documento principal, mediante \cmd  usepackage-, para poder
usar el estilo.

No voy  a explicar todos los  estilos adicionales, porque  ello sería tedioso,
sino exclusivamente los más conocidos e interesantes:

\begin{description}

\item[Apalike:] Este estilo  también fue escrito por Oren  Patashnik (el autor
  de  \btx-).   Es  un  estilo  de  tipo  ``Autor-año''  y  para  su  correcto
  funcionamiento, hay que cargar en  los documentos \ltx- el paquete del mismo
  nombre que el estilo: \ltr apalike-.

\item[Natbib:] Natbib es al mismo tiempo un estilo y un paquete, aunque quizás
  en él  destaque más la parte  de paquete, por la  cantidad de modificaciones
  que incorpora al funcionamiento del comando \cmd cite-.

  Durante mucho tiempo  (hasta que conocí el trabajo de  unos compañeros de la
  Universidad de  Murcia que han  diseñado \ltr flexbib-), he  considerado que
  \ltr  natbib- era  la mejor  opción.  Permite  elegir entre  citas  del tipo
  Autor-año  o citas numéricas.   Incluye versiones  adaptadas de  los estilos
  estándar  (salvo  el  estilo  \ltr  alpha-) llamadas  \ltr  plainnat-,  \ltr
  abbrvnat-  y  \ltr unsrtnat-;  añade  un campo  \ltr  usr-,  dispone de  una
  estupenda documentación e incorpora numerosas modificaciones y mejoras en el
  procedimiento de las citas.

\end{description}


\subsection{Cómo españolizar los estilos estándar}
\label{sec:como-espan-algun}

Los estilos  estándar de \btx- están  pensados para documentos que  se vayan a
redactar en inglés y por lo  tanto usan partículas, palabras y abreviaturas en
inglés. Para conseguir  que nuestra lista de referencias  figure totalmente en
español, sin palabras en inglés, existen varios procedimientos.

\begin{enumerate}

\item Lo más  cómodo es usar algún  estilo o paquete que añada  a los paquetes
  estándar la sensibilidad a \ltr babel-. Hay varios paquetes que hacen eso, y
  más  cosas.  Podemos citar  ``\texttt{abstyles-babel}'', diseñado  por Tomás
  Bautista y que es una adaptación  de los abstyles, para que sean sensibles a
  \ltr babel-, o los paquetes \ltr babelbib- y \ltr flexbib-.

\item Usar algún estilo diseñado para el idioma español. En Internet se pueden
  localizar  algunos, y  en la  CTAN está  disponible el  estilo ``spain.bst''
  basado en las indicaciones que sobre referencias bibliográficas se contienen
  en el ``Diccionario de ortografía técnica'' de J.  Martínez de Sousa.

\item Podemos  también generar nuestro propio  estilo bibliográfico, preparado
  para trabajar en  un idioma concreto.  Al respecto estúdiese  lo dicho en la
  sección \ref{sec:como-crear-estilos}.

\item Por  último, podemos modificar el  fichero donde se  contiene algún otro
  estilo, y  prepararlo para que  trabaje en español. Esta  última posibilidad
  exige un  mayor conocimiento del  funcionamiento interno de  \btx-. Conviene
  por  lo   tanto  aprender   muy  bien   lo  que  se   expone  en   la  parte
  \ref{part:btx-avanzado} de este documento.

\end{enumerate}

Para esta  parte de la  guía, referida al  nivel básico, el  procedimiento más
adecuado es el primero de los mencionados.

\clearpage

\part{\btx- intermedio}
\label{part:btx-intermedio}

\section{Generación automatizada de estilos bibliográficos}
\label{sec:como-crear-estilos}

Si ninguno  de los estilos bibliográficos  que acompañan a \btx-,  o que están
disponibles en  la CTAN  nos convence, podemos  generar nuestro  propio estilo
bibliográfico. Ello se  puede hacer escribiendo a mano  nuestro propio estilo,
lo que exige  un profundo conocimiento del lenguaje usado  por los ficheros de
estilo   (véase  la   parte  \ref{part:btx-avanzado}),   o   mediante  algunas
herramientas  dirigidas  a  la   generación  de  ficheros  \tpf  bst-.   Estas
herramientas, cuando  son ejecutadas, empiezan preguntando al  usuario por las
características  que debe  tener el  nuevo  estilo (preguntas  en inglés,  ¡of
course!), y,  con las  respuestas dadas,  se genera un  fichero \tpf  bst- que
contiene el  nuevo estilo  o, lo que  es más  normal, un fichero  intermedio a
partir del cual es posible generar el fichero \tpf bst-.

La herramienta  que voy  a analizar  aquí funciona de  esa segunda  manera: el
fichero intermedio tiene la extensión \tpf dbj- y la herramienta se llama \ltr
makebst- y  forma parte del paquete <<\texttt{custom-bib}>>  (disponible en la
CTAN).

\subsection{Makebst}
\label{sec:makebst}

Makebst es un  programa para \TeX{} diseñado por Patrick  W.  Daly que permite
generar ficheros bibliográficos personalizados.


\subsection{Uso de los ficheros \tpf dbj-}
\label{sec:uso-de-los}


\section{Paquetes \ltx- relacionados con \btx-}
\label{sec:paqu-ltx-relac}

En la CTAN  existen numerosos paquetes que añaden utilidades a  \btx- o que de
alguna manera amplían las posibilidades del sistema.

No me es posible ocuparme de todos, y por ello seleccionaré los que me parecen
más interesantes.

\subsection{Referencias  bibliográficas  completas   en  el  lugar  donde  son citadas}
\label{sec:refer-bibl-compl}

\subsection{Generación de varias listas bibliográficas}
\label{sec:el-paquete-bibunits}


\subsection{Otros paquetes}
\label{sec:otros-paquetes}


\section{Otras cuestiones de interés}
\label{sec:intr-comand-ltx}

\subsection{El comando newblock y la opción openbib}
\label{sec:la-opcion-de}

\subsection{Elementos @preamble en los ficheros \tpf bib-}
\label{sec:elementos-preamble}

\subsection{Comentarios y registros desconocidos en los ficheros \tpf bib-}
\label{sec:elementos-comment}


\clearpage

\part{\btx- avanzado}
\label{part:btx-avanzado}

\section{Los ficheros de estilo como programas}
\label{sec:los-ficheros-de}

Los ficheros  de estilo manejados  por \btx- son en  realidad \emph{programas}
escritos en un  lenguaje innominado que informan a \btx-  acerca de cómo deben
formatearse los  datos leídos de la base  de datos. Como el  lenguaje no tiene
nombre oficial, cuando me  tenga que referir a él lo denominaré  BST que es la
extensión utilizada  por los ficheros de estilo\footnote{Eso  no significa que
  el lenguaje se llame BST. Simplemente,  si no uso un nombre para referirme a
  él, en las próximas páginas tendría que emplear demasiados circunloquios.}.


\subsection{Cuestiones generales}
\label{sec:cuestiones-generales}

\subsubsection{Reglas generales de sintaxis}
\label{sec:reglas-generales-de}

Las características generales de BST son las siguientes:

\begin{enumerate}

\item No se  distingue entre mayúsculas y minúsculas  \emph{para los elementos
    del  lenguaje},  entendiendo  por  tales  los  nombres  de  los  comandos,
  variables y funciones,  así como las cadenas de texto  usadas como patrón en
  determinadas funciones  internas. Por el  contrario, en el contenido  de los
  campos y de  las variables de texto, sí se distinguen  las mayúsculas de las
  minúsculas.

\item Los saltos de línea y los tabuladores se tratan como espacios en blanco,
  y dos  o más  espacios en blanco  se tratan  como si fueran  sólo uno.   Y a
  diferencia  de otros  lenguajes  no existe  ningún  carácter reservado  para
  indicar el fin de  una instrucción: ni el punto y coma  (como en C o Pascal)
  ni el  salto de línea (como  en Basic y  similares). Los saltos de  línea se
  usan exclusivamente  para hacer más  comprensible para los seres  humanos lo
  que una función hace.

  No obstante lo anterior, de cara  a la información que \btx- emite cuando se
  produce algún error (y que se almacena en el fichero \tpf blg- generado tras
  la ejecución de \btx-), se recomienda:

  \begin{enumerate}

  \item  Insertar una  o  más líneas  en  blanco entre  la  definición de  dos
    funciones.

  \item No introducir líneas en blanco dentro de una función.

  \end{enumerate}

  Lo anterior,  por supuesto, no  se aplica al  contenido de las  variables de
  texto. En ellas  todos los espacios en blanco son  relevantes y se distingue
  entre espacios  en blanco  y tabuladores\footnote{Ello con  independencia de
    que si  finalmente el contenido de  la variable termina  formando parte de
    una lista de referencias insertada en un documento \tpf tex-, los espacios
    en  blanco y  tabuladores existentes  en ella  reciban el  tratamiento que
    \ltx- da  a tales caracteres.}. Asimismo  las constantes de  texto, que se
  introducen entrecomilladas,  deben escribirse en  la misma línea:  entre las
  comillas de apertura y cierre de una cadena de texto no puede haber un salto
  de línea.

\item Se puede hacer uso de las llaves para agrupar bloques de código.

\item El carácter  ``\%'' está reservado para los comentarios:  Todo lo que se
  escriba a la derecha del mismo, hasta el final de la línea, será ignorado.

\end{enumerate}

\subsubsection{Estructura de los programas BST}
\label{sec:estructura-de-un}

\subsection{Objetos del lenguaje BST}
\label{sec:sintaxis-de-los}

\subsubsection{La lista de datos}
\label{sec:la-lista-de-2}

\subsubsection{Variables y constantes}
\label{sec:almac-de-los}

\subsubsection{Funciones}
\label{sec:funciones}

\section{Conclusión: Otros usos de \btx-}
\label{sec:otros-usos-de}

\btx- fue diseñado para la gestión de la bibliografía. Pero la flexibilidad de
formato de  los ficheros \tpf bib-, y  la potencia de los  ficheros de estilo,
permiten extender sus posibilidades hasta  el punto de que podamos usarlo para
muchas otras tareas.  Podemos considerar  a \btx- como una herramienta que nos
permite insertar en  un documento información procedente de  una base de datos
externa, así como decidir de qué manera debe formatearse tal información.

En \cite{cascales00} se explica cómo usar \btx- para generar una base de datos
de problemas  matemáticos. Y  en \cite{markey05} se  habla del posible  uso de
\btx- para  generar una lista de  publicaciones y una  libreta de direcciones.
Díez de Arriba y Javier Bezos, por su parte, han diseñado el paquete Gloss que
usa  \btx-  para generar  glosarios,  mientras  que yo,  que  me  dedico a  la
enseñanza del Derecho,  he usado \btx- para con una base  de datos de exámenes
tipo test,  y, en  publicaciones de tipo  jurídico, con  una base de  datos de
sentencias, de tal  modo que un simple comando \cmd  cite- pudiera insertar en
el texto un rótulo identificativo  de una sentencia (del tipo <<STS 21-4-56>>)
y, al final del documento, una lista con las sentencias citadas.

Estos usos \emph{alternativos} de \btx- pueden requerir un \emph{paquete} para
implementarlos. Sobre  todo si se pretende que  su uso no impida,  en el mismo
documento, el uso de \btx- al modo  \emph{normal}. Así ocurre en el caso de la
generación  de glosarios.   O pueden  exigir  el diseño  de un  tipo nuevo  de
registro y  una función que  lo formatee incorporada  a alguno de  los estilos
estándar de \btx- (como hice para la  base de datos de sentencias y para la de
exámenes  tipo  test), o  simplemente  un  uso  determinado de  los  registros
normales.

Lo importante es que tengamos claro que \btx- esconde muchas más posibilidades
de lo que a primera vista pudiera parecer.




\appendix


\section*{Apéndice  A:  GNU  Free  Documentation License  (Licencia  GNU  para
  documentación)}
\label{sec:gnu-free-docum}
\addcontentsline{toc}{section}{Apéndice  A.  GNU  Free  Documentation License}
% CAMBIO DE IDIOMA: INICIO
\selectlanguage{english} \small

\begin{center}
   
  Version 1.3, 3 November 2008
   
  Copyright \copyright 2000,2001,2002,2007,2008 Free Software Foundation, Inc. <<\url{https://fsf.org/}>>
   
  Everyone is permitted to copy and distribute verbatim copies of this license
  document, but changing it is not allowed.
\end{center}

\begin{center}
  {\bf\large Preamble}
\end{center}

The purpose of this License is to make a manual, textbook, or other functional and useful document "free" in the sense of freedom: to assure everyone the effective freedom to copy and redistribute it, with or without modifying it, either commercially or noncommercially. Secondarily, this License preserves for the author and publisher a way to get credit for their work, while not being considered responsible for modifications made by others.

This License is a kind of "copyleft", which means that derivative works of the document must themselves be free in the same sense. It complements the GNU General Public License, which is a copyleft license designed for free software.

We have designed this License in order to use it for manuals for free software, because free software needs free documentation: a free program should come with manuals providing the same freedoms that the software does. But this License is not limited to software manuals; it can be used for any textual work, regardless of subject matter or whether it is published as a printed book. We recommend this License principally for works whose purpose is instruction or reference.

\begin{center}
  {\Large\bf 1. APPLICABILITY AND DEFINITIONS}
\end{center}

\begin{center}
  {\Large\bf 2. VERBATIM COPYING}
\end{center}

\begin{center}
  {\Large\bf 3. COPYING IN QUANTITY}
\end{center}


\begin{center}
  {\Large\bf 4. MODIFICATIONS}
\end{center}


\begin{center}
  {\Large\bf 5. COMBINING DOCUMENTS}
\end{center}


\begin{center}
  {\Large\bf 6. COLLECTIONS OF DOCUMENTS}
\end{center}


\begin{center}
  {\Large\bf 7. AGGREGATION WITH INDEPENDENT WORKS}
\end{center}



\begin{center}
  {\Large\bf 8. TRANSLATION}
\end{center}



\begin{center}
  {\Large\bf 9. TERMINATION}
\end{center}


\begin{center}
  {\Large\bf 10. FUTURE REVISIONS OF THIS LICENSE}
\end{center}


\begin{center}
  {\Large\bf ADDENDUM: How to use this License for your documents}
\end{center}

To use this License in a document you have written, include a copy of
the License in the document and put the following copyright and
license notices just after the title page:

\bigskip
\begin{quote}
  Copyright  \copyright  YEAR  YOUR NAME.
  Permission is granted to copy, distribute and/or modify this document
  under the terms of the GNU Free Documentation License, Version 1.3
  or any later version published by the Free Software Foundation;
  with no Invariant Sections, no Front-Cover Texts, and no Back-Cover Texts.
  A copy of the license is included in the section entitled 
  "GNU Free Documentation License".
\end{quote}
\bigskip

If you have Invariant Sections, Front-Cover Texts and Back-Cover Texts,
replace the "with...Texts." line with this:

\bigskip
\begin{quote}
  with the Invariant Sections being LIST THEIR TITLES, with the
  Front-Cover Texts being LIST, and with the Back-Cover Texts being LIST.
\end{quote}
\bigskip

If you have Invariant Sections without Cover Texts, or some other combination 
of the three, merge those two alternatives to suit the situation.

If your document contains nontrivial examples of program code, we recommend 
releasing these examples in parallel under your choice of free software 
license, such as the GNU General Public License, to permit their use in 
free software. 

% CAMBIO DE IDIOMA: FIN
\selectlanguage{spanish}



%REFERENCIAS EN DOS ESTILOS
%Solo se listan las referencias invocadas en el documento

% Estilo APA
\bibliography{biblio}\bibliographystyle{apacite}
% EStilo PERSONALIZADO
%\bibliography{biblio}\bibliographystyle{articulo}

\end{document}
